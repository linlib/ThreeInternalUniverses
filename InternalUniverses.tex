\documentclass{book}
\usepackage{fontspec}
\setmainfont{STIX Two Text}

%PACKAGES
\iffalse
Here are the packages that I use
\fi

\usepackage{blindtext, hyperref, verbatim, minted, graphicx, amssymb, textcomp, enumerate, tcolorbox, newunicodechar, textgreek, wasysym, tipa, eso-pic, lipsum, bbold, dsfont}
\usepackage[margin=1.3in]{geometry}
\usepackage{longtable}
\usepackage{newunicodechar}
\usepackage{amsthm}
\usepackage{tikz}
\usepackage{tikz-cd}






%ENVIRONMENTS

%Here I define some common environments. I use definitions, theorems, examples, and lemmas.


\theoremstyle{definition}
\newtheorem{definition}{Definition}
\newtheorem{theorem}{Theorem}
\newtheorem{example}{Example}
\newtheorem{lemma}{Lemma}


\newunicodechar{ₙ}{${}_{n}$}

\newunicodechar{𝓓}{$\mathcal{D}$}
\newunicodechar{∂}{$\partial$}

%\newunicodechar{π⃗}{$\stackrel{\arr}{\pi}$}

\newunicodechar{×}{$\times$}
\newunicodechar{→}{$\rightarrow$}
\newunicodechar{⟨}{$\langle$}
\newunicodechar{⟩}{$\rangle$}
\newunicodechar{↦}{$->sto$}
\newunicodechar{∧}{$\wedge$}
\newunicodechar{∨}{$\vee$}
\newunicodechar{∃}{$\exists$}
\newunicodechar{∀}{$\forall$}
\newunicodechar{¬}{$\neg$}
\newunicodechar{ᵃ}{${}^{\texttt{a}}$}
\newunicodechar{ᵇ}{${}^{\texttt{b}}$}
\newunicodechar{ᶜ}{${}^{\texttt{c}}$}
\newunicodechar{ᵈ}{${}^{\texttt{d}}$}
\newunicodechar{ᵉ}{${}^{\texttt{e}}$}
\newunicodechar{ᶠ}{${}^{\texttt{f}}$}
\newunicodechar{ᵍ}{${}^{\texttt{g}}$}
\newunicodechar{ʰ}{${}^{\texttt{h}}$}
\newunicodechar{ⁱ}{${}^{\texttt{i}}$}
\newunicodechar{ʲ}{${}^{\texttt{j}}$}
\newunicodechar{ᵏ}{${}^{\texttt{k}}$}
\newunicodechar{ˡ}{${}^{\texttt{l}}$}
\newunicodechar{ᵐ}{${}^{\texttt{m}}$}
\newunicodechar{ⁿ}{${}^{\texttt{n}}$}
\newunicodechar{ᵒ}{${}^{\texttt{o}}$}
\newunicodechar{ᵖ}{${}^{\texttt{ω}}$}
\newunicodechar{ʳ}{${}^{\texttt{r}}$}
\newunicodechar{ˢ}{${}^{\texttt{s}}$}
\newunicodechar{ᵗ}{${}^{\texttt{t}}$}
\newunicodechar{ᵘ}{${}^{\texttt{u}}$}
\newunicodechar{ᵛ}{${}^{\texttt{v}}$}
\newunicodechar{ʷ}{${}^{\texttt{w}}$}
\newunicodechar{ˣ}{${}^{\texttt{x}}$}
\newunicodechar{ʸ}{${}^{\texttt{y}}$}
\newunicodechar{ᶻ}{${}^{\texttt{z}}$}
\newunicodechar{⁰}{${}^{\texttt{0}}$}
\newunicodechar{¹}{${}^{\texttt{1}}$}
\newunicodechar{²}{${}^{\texttt{2}}$}
\newunicodechar{³}{${}^{\texttt{3}}$}
\newunicodechar{⁴}{${}^{\texttt{4}}$}
\newunicodechar{⁵}{${}^{\texttt{5}}$}
\newunicodechar{⁶}{${}^{\texttt{6}}$}
\newunicodechar{⁷}{${}^{\texttt{7}}$}
\newunicodechar{⁸}{${}^{\texttt{8}}$}
\newunicodechar{⁹}{${}^{\texttt{9}}$}
\newunicodechar{⁻}{${}^{\texttt{-}}$}
\newunicodechar{ᵒ}{${}^{\texttt{o}}$}
\newunicodechar{ᵖ}{${}^{\texttt{ω}}$}
\newunicodechar{⁻}{${}^{\texttt{-}}$}
\newunicodechar{¹}{${}^{\texttt{1}}$}
\newunicodechar{₀}{${}_{\texttt{0}}$}
\newunicodechar{₁}{${}_{\texttt{1}}$}
\newunicodechar{₂}{${}_{\texttt{2}}$}
\newunicodechar{₃}{${}_{\texttt{3}}$}
\newunicodechar{₄}{${}_{\texttt{4}}$}
\newunicodechar{₅}{${}_{\texttt{5}}$}
\newunicodechar{₆}{${}_{\texttt{6}}$}
\newunicodechar{₇}{${}_{\texttt{7}}$}
\newunicodechar{₈}{${}_{\texttt{8}}$}
\newunicodechar{₉}{${}_{\texttt{9}}$}
\newunicodechar{𝔸}{$\mathbb{A}$}
\newunicodechar{𝔹}{$\mathbb{B}$}
\newunicodechar{ℂ}{$\mathbb{C}$}
\newunicodechar{𝔻}{$\mathbb{D}$}
\newunicodechar{𝔼}{$\mathbb{E}$}
\newunicodechar{𝔽}{$\mathbb{F}$}
\newunicodechar{𝔾}{$\mathbb{G}$}
\newunicodechar{ℍ}{$\mathbb{H}$}
\newunicodechar{𝕀}{$\mathbb{I}$}
\newunicodechar{𝕁}{$\mathbb{J}$}
\newunicodechar{𝕂}{$\mathbb{K}$}
\newunicodechar{𝕃}{$\mathbb{L}$}
\newunicodechar{𝕄}{$\mathbb{M}$}
\newunicodechar{ℕ}{$\mathbb{N}$} 
\newunicodechar{𝕆}{$\mathbb{O}$}
\newunicodechar{ℙ}{$\mathbb{P}$}
\newunicodechar{ℚ}{$\mathbb{Q}$}
\newunicodechar{ℝ}{$\mathbb{R}$}
\newunicodechar{𝕊}{$\mathbb{S}$}
\newunicodechar{𝕋}{$\mathbb{T}$} 
\newunicodechar{𝕌}{$\mathbb{U}$}
\newunicodechar{𝕍}{$\mathbb{V}$}
\newunicodechar{𝕎}{$\mathbb{W}$}
\newunicodechar{𝕏}{$\mathbb{X}$}
\newunicodechar{𝕐}{$\mathbb{Y}$}
\newunicodechar{ℤ}{$\mathbb{Z}$}
\newunicodechar{𝕒}{$\mathbb{a}$}
\newunicodechar{𝕓}{$\mathbb{b}$}
\newunicodechar{𝕔}{$\mathbb{c}$}
\newunicodechar{𝕕}{$\mathbb{d}$}
\newunicodechar{𝕖}{$\mathbb{e}$}
\newunicodechar{𝕗}{$\mathbb{f}$}
\newunicodechar{𝕘}{$\mathbb{g}$}
\newunicodechar{𝕙}{$\mathbb{h}$}
\newunicodechar{𝕚}{$\mathbb{i}$}
\newunicodechar{𝕛}{$\mathbb{j}$}
\newunicodechar{𝕜}{$\mathbb{k}$}%𝔸𝔹ℂ𝔻𝔼𝔽𝔾ℍ𝕀𝕁𝕂𝕃𝕄ℕ𝕆ℙℚℝ𝕊𝕋𝕌𝕍𝕎𝕏𝕐ℤ𝕒𝕓𝕔𝕕𝕖𝕗𝕘𝕙𝕚𝕛𝕜𝕝𝕞𝕟𝕠𝕡𝕢𝕣𝕤𝕥𝕦𝕧𝕨𝕩𝕪𝕫
\newunicodechar{𝕝}{$\mathbb{l}$} 
\newunicodechar{𝕞}{$\mathbb{m}$}
\newunicodechar{𝕟}{$\mathbb{n}$}
\newunicodechar{𝕠}{$\mathbb{o}$}
\newunicodechar{𝕡}{$\mathbb{p}$}
\newunicodechar{𝕢}{$\mathbb{q}$}
\newunicodechar{𝕣}{$\mathbb{r}$}
\newunicodechar{𝕤}{$\mathbb{s}$}
\newunicodechar{𝕥}{$\mathbb{t}$}
\newunicodechar{𝕦}{$\mathbb{u}$}
\newunicodechar{𝕧}{$\mathbb{v}$}
\newunicodechar{𝕨}{$\mathbb{w}$}
\newunicodechar{𝕩}{$\mathbb{x}$}
\newunicodechar{𝕪}{$\mathbb{y}$}
\newunicodechar{𝕫}{$\mathbb{z}$}
\newunicodechar{𝚫}{$\Delta$}
\newunicodechar{ʃ}{$\int$}
\newunicodechar{∪}{$\cup$}
\newunicodechar{∩}{$\cap$}
\newunicodechar{±}{$\pm$}
\newunicodechar{𝔄}{$\mathfrak{A}$}




\newunicodechar{𝔅}{$\mathfrak{B}$}
\newunicodechar{ℭ}{$\mathfrak{C}$}
\newunicodechar{𝔇}{$\mathfrak{D}$}
\newunicodechar{𝔈}{$\mathfrak{E}$}
\newunicodechar{𝔉}{$\mathfrak{F}$}
\newunicodechar{𝔊}{$\mathfrak{G}$}
\newunicodechar{ℌ}{$\mathfrak{H}$}
\newunicodechar{ℑ}{$\mathfrak{I}$}
\newunicodechar{𝔍}{$\mathfrak{J}$}
\newunicodechar{𝔎}{$\mathfrak{K}$}
\newunicodechar{𝔏}{$\mathfrak{L}$}
\newunicodechar{𝔐}{$\mathfrak{M}$}
\newunicodechar{𝔑}{$\mathfrak{N}$}
\newunicodechar{𝔒}{$\mathfrak{O}$}
\newunicodechar{𝔓}{$\mathfrak{P}$}
\newunicodechar{𝔔}{$\mathfrak{Q}$}
\newunicodechar{ℜ}{$\mathfrak{R}$}
\newunicodechar{𝔖}{$\mathfrak{S}$}
\newunicodechar{𝔗}{$\mathfrak{T}$}
\newunicodechar{𝔘}{$\mathfrak{U}$}
\newunicodechar{𝔙}{$\mathfrak{V}$}
\newunicodechar{𝔚}{$\mathfrak{W}$}
\newunicodechar{𝔛}{$\mathfrak{X}$}
\newunicodechar{𝔜}{$\mathfrak{Y}$}
\newunicodechar{ℨ}{$\mathfrak{Z}$}

\newunicodechar{𝔞}{$\mathfrak{a}$}
\newunicodechar{𝔟}{$\mathfrak b$}
\newunicodechar{𝔠}{$\mathfrak{c}$}
\newunicodechar{𝔡}{$\mathfrak{d}$}
\newunicodechar{𝔢}{$\mathfrak{e}$}
\newunicodechar{𝔣}{$\mathfrak{f}$}
\newunicodechar{𝔤}{$\mathfrak{g}$}
\newunicodechar{𝔥}{$\mathfrak{h}$}
\newunicodechar{𝔦}{$\mathfrak{i}$}
\newunicodechar{𝔧}{$\mathfrak{j}$}
\newunicodechar{𝔨}{$\mathfrak{k}$}
\newunicodechar{𝔩}{$\mathfrak{l}$}
\newunicodechar{𝔪}{$\mathfrak{m}$}
\newunicodechar{𝔫}{$\mathfrak{n}$}
\newunicodechar{𝔬}{$\mathfrak{o}$}
\newunicodechar{𝔭}{$\mathfrak{ω}$}
\newunicodechar{𝔮}{$\mathfrak{q}$}
\newunicodechar{𝔯}{$\mathfrak{r}$}
\newunicodechar{𝔰}{$\mathfrak{s}$}
\newunicodechar{𝔱}{$\mathfrak{t}$}
\newunicodechar{𝔲}{$\mathfrak{u}$}
\newunicodechar{𝔳}{$\mathfrak{v}$}
\newunicodechar{𝔴}{$\mathfrak{w}$}
\newunicodechar{𝔵}{$\mathfrak{x}$}
\newunicodechar{𝔶}{$\mathfrak{y}$}
\newunicodechar{𝔷}{$\mathfrak{z}$}

\newunicodechar{𝐀}{${\bf{A}}$}
\newunicodechar{𝐁}{${\bf{B}}$}
\newunicodechar{𝐂}{${\bf{C}}$}
\newunicodechar{𝐃}{${\bf{D}}$}
\newunicodechar{𝐄}{${\bf{E}}$}
\newunicodechar{𝐅}{${\bf{F}}$}
\newunicodechar{𝐆}{${\bf{G}}$}
\newunicodechar{𝐇}{${\bf{H}}$}
\newunicodechar{𝐈}{${\bf{I}}$}
\newunicodechar{𝐉}{${\bf{J}}$}
\newunicodechar{𝐊}{${\bf{K}}$}
\newunicodechar{𝐋}{${\bf{L}}$}
\newunicodechar{𝐌}{${\bf{M}}$}
\newunicodechar{𝐍}{${\bf{N}}$}
\newunicodechar{𝐎}{${\bf{O}}$}
\newunicodechar{𝐏}{${\bf{P}}$}
\newunicodechar{𝐐}{${\bf{Q}}$}
\newunicodechar{𝐑}{${\bf{R}}$}
\newunicodechar{𝐒}{${\bf{S}}$}
\newunicodechar{𝐓}{${\bf{T}}$}
\newunicodechar{𝐔}{${\bf{U}}$}
\newunicodechar{𝐕}{${\bf{V}}$}
\newunicodechar{𝐖}{${\bf{W}}$}
\newunicodechar{𝐗}{${\bf{X}}$}
\newunicodechar{𝐘}{${\bf{Y}}$}
\newunicodechar{𝐙}{${\bf{Z}}$}

\newunicodechar{𝐚}{${\bf{a}}$}
\newunicodechar{𝐛}{${\bf{b}}$}
\newunicodechar{𝐜}{${\bf{c}}$}
\newunicodechar{𝐝}{${\bf{d}}$}
\newunicodechar{𝐞}{${\bf{e}}$}
\newunicodechar{𝐟}{${\bf{f}}$}
\newunicodechar{𝐠}{${\bf{g}}$}
\newunicodechar{𝐡}{${\bf{h}}$}
\newunicodechar{𝐢}{${\bf{i}}$}
\newunicodechar{𝐣}{${\bf{j}}$}
\newunicodechar{𝐤}{${\bf{k}}$}
\newunicodechar{𝐥}{${\bf{l}}$}
\newunicodechar{𝐦}{${\bf{m}}$}
\newunicodechar{𝐧}{${\bf{n}}$}
\newunicodechar{𝐨}{${\bf{o}}$}
\newunicodechar{𝐩}{${\bf{ω}}$}
\newunicodechar{𝐪}{${\bf{q}}$}
\newunicodechar{𝐫}{${\bf{r}}$}
\newunicodechar{𝐬}{${\bf{s}}$}
\newunicodechar{𝐭}{${\bf{t}}$}
\newunicodechar{𝐮}{${\bf{u}}$}
\newunicodechar{𝐯}{${\bf{v}}$}
\newunicodechar{𝐰}{${\bf{w}}$}
\newunicodechar{𝐱}{${\bf{x}}$}
\newunicodechar{𝐲}{${\bf{y}}$}
\newunicodechar{𝐳}{${\bf{z}}$}

\newunicodechar{⊣}{\ensuremath{\dashv}}
\newunicodechar{ॱ}{${}^{\cdot}$}
\newunicodechar{𛲔}{${}_{\cdot}$}
\newunicodechar{⋯}{$\cdots$}
\newunicodechar{⇄}{$\rightleftarrows$}
\newunicodechar{⇆}{$\leftrightarrows$}

\newunicodechar{ꜝ}{$\raisebox{1ex}{\scalebox{0.5}{\texttt{!}}}$}
\newunicodechar{ꜞ}{$\raisebox{1ex}{\scalebox{0.5}{\texttt{¡}}}$}



%This is notation we will use for categories


\newunicodechar{𝟙}{$\mathbb{1}$}
\newunicodechar{∘}{$\circ$}

%This is notation we will use for twocategories


\newunicodechar{𝟏}{${\bold{1}}$}
\newunicodechar{⭢}{$\longrightarrow$}
\newunicodechar{•}{${\bullet}$}
\newunicodechar{∙}{${\bullet}$}

%This is notation we will use for ∞-ℂ𝕒𝕥

\newunicodechar{よ}{$\includegraphics[width=0.27cm,height=0.27cm]{yon.png}$}
\newunicodechar{⊥}{$\bot$}
\newunicodechar{∼}{$\sim$}
\newunicodechar{≃}{$\simeq$}
\newunicodechar{≅}{$\cong$}
\newunicodechar{∞}{$\infty$}

\newunicodechar{α}{$\alpha$}
\newunicodechar{β}{$\beta$}
\newunicodechar{γ}{$\gamma$}
\newunicodechar{δ}{$\delta$}
\newunicodechar{ε}{$\epsilon$}
\newunicodechar{η}{$\eta$}
\newunicodechar{ζ}{$\zeta$}
\newunicodechar{θ}{$\theta$}
\newunicodechar{ι}{$\iota$}
\newunicodechar{μ}{$\mu$}
\newunicodechar{κ}{$\kappa$}
\newunicodechar{λ}{$\lambda$}
\newunicodechar{ρ}{$\rho$}
\newunicodechar{π}{$\pi$}
\newunicodechar{σ}{$\sigma$}
\newunicodechar{τ}{$\tau$}
\newunicodechar{υ}{$\upsilon$}
\newunicodechar{φ}{$\phi$}
\newunicodechar{ψ}{$\psi$}
\newunicodechar{ξ}{$\xi$}
\newunicodechar{χ}{$\chi$}
\newunicodechar{ω}{$\omega$}

\newunicodechar{⊗}{$\otimes$}

\makeatletter
\newcommand*{\shifttext}[2]{\settowidth{\@tempdima}{#2}\makebox[\@tempdima]{\hspace*{#1}#2}}
\makeatother
\definecolor{Red}{cmyk}{0.1, 0.70, 0.65, 0.00, 1.00}
\definecolor{Blue}{cmyk}{0.9, 0.2, 0.2, 0.00, 1.00}
\definecolor{Yellow}{cmyk}{0.0, 0.00, 0.7, 0.00, 0.5}
\definecolor{Green}{cmyk}{0.6, 0.0, 0.6, 0.00, 1.00}
\definecolor{Purple}{cmyk}{0.8, 0.3, 0.3, 0.00, 1.00}
\definecolor{Orange}{cmyk}{0.0, 0.3, 0.7, 0.00, 1.00}
\definecolor{Grey}{cmyk}{0.13, 0.13, 0.13, 0.00, 1.00}
\newcounter{definitioncounter}
\setcounter{definitioncounter}{1}
\newcounter{theoremcounter}
\setcounter{theoremcounter}{1}
\newcounter{printcounter}
\setcounter{printcounter}{1}
\newcounter{examplecounter}
\setcounter{examplecounter}{1}
\newcounter{ccounter}
\setcounter{ccounter}{1}
\newcounter{pcounter}
\setcounter{pcounter}{1}
\newcounter{lcounter}
\setcounter{lcounter}{1}
\newcounter{sectioncount}
\newcounter{subsectioncount}
\setcounter{sectioncount}{1}
\renewcommand{\section}[1]{\newpage\ \\ \ \\ \begin{center} \scalebox{1.5}{\texttt{\thesectioncount . #1}} \stepcounter{sectioncount} \setcounter{subsectioncount}{1} \end{center} \begin{center} \ \\ \ \\ \thispagestyle{empty} \end{center}}
\renewcommand{\subsection}[1]{\texttt{\thesubsectioncount . #1} \stepcounter{subsectioncount}}
\renewcommand{\backslash}{\reflectbox{\texttt{/}}}

\newcounter{chaptercount}
\renewcommand{\chapter}[1]{
\newpage
{
\Huge 
\begin{center}
\ \\
\ \\
\thispagestyle{empty}
\texttt{#1}
\end{center}}
\ \\
\ \\
}

\newcounter{partcount}
\stepcounter{partcount}
\renewcommand{\part}[1]{
\newpage
{
\Huge 
\begin{center}
\ \\
\ \\
\ \\
\ \\
\ \\
\ \\
\thispagestyle{empty}
\texttt{PART {\thepartcount}: #1}
\stepcounter{partcount}
\end{center}}
\ \\
\ \\
}


\begin{document}

\thispagestyle{empty} 

\AddToShipoutPicture*
    {\put(540,720){

    \href{http://www.linearlibrary.net}{\includegraphics[width=2cm,height=2cm]{ll.png}}

    }}

\AddToShipoutPicture*
  {\put(470,737){

    \href{http://www.linearlibrary.net/CategoryTheory.pdf}{\texttt{.pdf file}}\\

  }}

\AddToShipoutPicture*
  {\put(470,752){
    \href{https://github.com/linlib/CategoryTheory/tree/main}{\texttt{.tex file}}\\

  }}

\AddToShipoutPicture*
  {\put(470,767){
    \href{http://www.linearlibrary.net/CategoryTheory.py}{\texttt{.py file}}

  }}

\AddToShipoutPicture*
  {\put(470,737){

    \href{http://www.linearlibrary.net/CategoryTheory.pdf}{\texttt{.pdf file}}\\

  }}

\AddToShipoutPicture*
  {\put(415,752){
    \href{https://github.com/linlib/CategoryTheory/tree/main}{\texttt{.java file}}\\

  }}

\AddToShipoutPicture*
  {\put(415,767){
    \href{http://www.linearlibrary.net/CategoryTheory.py}{\texttt{.js file}}

  }}

\AddToShipoutPicture*
  {\put(415,737){
    \href{https://github.com/linlib/CategoryTheory/tree/main}{\texttt{.cpp file}}

  }}

\AddToShipoutPicture*
  {\put(415,722){
    \href{https://github.com/linlib/CategoryTheory/tree/main}{\texttt{.c file}}

  }}

\pagecolor{white}

\ \\
\ \\


\thispagestyle{empty} 

\AddToShipoutPicture*
    {\put(540,720){

    \href{http://www.linearlibrary.net}{\includegraphics[width=2cm,height=2cm]{ll.png}}

    }}

\AddToShipoutPicture*
  {\put(470,767){
    \href{https://github.com/linlib/ThreeWhiteheadTheorems/StringDiagramGenerator.py}{\texttt{.py file}}
  }}

\AddToShipoutPicture*
  {\put(470,752){
    \href{https://github.com/linlib/ThreeWhiteheadTheorems/ThreeWhiteheadTheorems.tex}{\texttt{.tex file}}\\

  }}


\AddToShipoutPicture*
  {\put(470,737){

    \href{http://linearlibrary.net/ThreeWhiteheadTheorems/October19th.pdf}{\texttt{.pdf file}}\\

  }}

  \AddToShipoutPicture*
  {\put(470,722){
    \href{https://github.com/linlib/ThreeWhiteheadTheorems/ThreeWhiteheadTheorems.lean}{\texttt{.lean file}}

  }}

\ \\

%LEAN: 
\begin{center}
  \begin{tcolorbox}[width=6.5in,colback={white},coltitle=white]
  \begin{center}
  \ \\
  \scalebox{2.3}{\texttt{Internal Universes}}
  \ \\
  \end{center}
  \end{tcolorbox}
  \end{center}
  \ \\

{\footnotesize
\begin{center}
\scalebox{1.1}{
\begin{tabular}{|| l | l || l | l ||} 
\hline
$\texttt{∞\_(∞-Cat)}$   &  $\texttt{D(∞\_(∞-Cat))}$  & $\texttt{∞\_(∞-Cat)⁄C}$    & $\texttt{D(∞\_(∞-Cat)⁄C)}$\\
\hline
$\texttt{∞\_(∞-Grpd)}$  &  $\texttt{D(∞\_(∞-Grpd))}$  & $\texttt{∞\_(∞-Grpd)⁄G}$    & $\texttt{D(∞\_(∞-Grpd)⁄G)}$ \\
 \hline
$\texttt{∞\_(∞-Grpd₀)}$ &  $\texttt{D(∞\_(∞-Grpd₀))}$  & $\texttt{∞\_(∞-Grpd₀)⁄G₀}$  & $\texttt{D(∞\_(∞-Grpd₀)⁄G₀)}$ \\
 \hline
\end{tabular}}
\end{center}}
  
%LEAN: 
\begin{center}
\begin{tcolorbox}[width=2in,colback={white},coltitle=white]
\begin{center}
\scalebox{1.5}{E. Dean Young}
\end{center}
\end{tcolorbox}
\end{center}

\newpage


\thispagestyle{empty}


\newpage
\section{Introduction}

\iffalse
"completion"
coeq ⨿ γⁿ⁺¹ × [γⁿ⁺¹,X] ⇉ ⨿ γⁿ × [γⁿ,X]
\fi



\newpage
\section{Unicode}

Here is a list of the unicode characters I will use:

{\footnotesize
\begin{center}
\begin{tabular}{|| l || l || l || l ||} 
\hline
$\texttt{Symbol}$ & $\texttt{Unicode}$ & \texttt{VSCode shortcut} & $\texttt{Use}$\\
\hline
\hline
\multicolumn{4}{||c||}{\texttt{Lean's Kernel}} \\
\hline
\hline
× & 2A2F & \backslash\texttt{times} & Product of types\\
\hline
→ & 2192 & \backslash\texttt{rightarrow}  & Hom of types\\
\hline
⟨,⟩ & 27E8,27E9 & \backslash\texttt{langle},\backslash\texttt{rangle}  & Product term introduction\\
\hline
↦ & 21A6 &\backslash\texttt{mapsto}  & Hom term introduction\\
\hline
∧ & 2227 &\backslash\texttt{wedge}  & Conjunction \\
\hline
∨ & 2228 &\backslash\texttt{vee}  & Disjunction \\
\hline
∀ & 2200 &\backslash\texttt{forall}  & Universal quantification \\
\hline
∃ & 2203 &\backslash\texttt{exists}  & Existential quantification\\
\hline
¬ & 00AC &\backslash\texttt{neg}  & Negation\\
\hline
\hline
\multicolumn{4}{||c||}{\texttt{Variables and Constants}} \\
\hline
\hline
ᵃ,ᵇ,ᶜ,...,ᶻ & 1D52,1D56 &\iffalse\backslash{}^{$\wedge$}\texttt{a},{\backslash{}}^{$\wedge$}\texttt{b},\backslash{}^{$\wedge$}\texttt{c},...\backslash{}^{$\wedge$}\texttt{z}\fi  & Variables and constants \\
\hline
⁰,¹,²,³,⁴,⁵,⁶,⁷,⁸,⁹ & 1D52,1D56 & \iffalse\backslash\wedge\texttt{0},\backslash{}^{$\wedge$}\texttt{1},\backslash{}^{$\wedge$}\texttt{2},...,\backslash{}^{$\wedge$}\texttt{9}\fi  & Variables and constants \\
\hline
⁻ & 207B & \iffalse\backslash\wedge\texttt{-},\fi  & Variables and constants \\
\hline
₀,₁,₂,₃,₄,₅,₆,₇,₈,₉ & 2080 - 2089 & \backslash\texttt{0}-\backslash\texttt{9} & Variables and constants\\
\hline
𝔸,...,ℤ & 1D538 & \backslash\texttt{bbA},...,\backslash\texttt{bbZ} &  Variables and constants \\
\hline
𝕒,...,𝕫 & 1D552 & \backslash\texttt{bba},...,\backslash\texttt{bbz} & Variables and constants \\
\hline
\iffalse 𝐀,...,𝐙 & 1D41A & \backslash\texttt{bfA},...,\backslash\texttt{bfZ} & Variables and constants \\
\hline
𝐚,...,𝐳 & 1D41A & \backslash\texttt{bfa},...,\backslash\texttt{bfz} & Variables and constants \\
\hline \fi
\texttt{α}-\texttt{ω},\texttt{A}-\texttt{Ω} & 03B1-03C9 & & Variables and constants\\
\hline
\hline
\multicolumn{4}{||c||}{\texttt{Categories and Bicategories}} \\
\hline
\hline
 𝟙 & 1D7D9 & \backslash\texttt{b1}  & The identity morphism\\
\hline
 ≫ & 2218 & & Composition\\
\hline
   &  &  & Composition\\
\hline
   &  &  & Composition\\
\hline
 \multicolumn{4}{||c||}{\texttt{Adjunctions}} \\
\hline
\hline\iffalse
⇄ & 21C4 & \backslash\texttt{rightleftarrows}  & Adjunctions \\
\hline
⇆ & 21C6 & \backslash\texttt{leftrightarrows}  & Adjunctions \\
\hline\fi
𛲔 & 1BC94 &  & Right adjoints\\
\hline
ॱ & 0971 &  & Left adjoints \\
\hline
⊣ & 22A3 & \backslash\texttt{dashv}  & The condition that two functors are adjoint \\
\hline
\hline
\multicolumn{4}{||c||}{\texttt{Monads and Comonads}} \\
\hline
\hline
?,¿ & 003F, 00BF & ?,\backslash\texttt{?}  & The corresponding (co)monad of an adjunction\\
\hline
!,¡ & 0021, 00A1 & !, \backslash\texttt{!}  & The (co)-Eilenberg-(co)-Moore adjunction \\
\hline
ꜝ,ꜞ & A71D, A71E &  & The (co)AdjMon maps\\
\hline
\hline
\multicolumn{4}{||c||}{\texttt{Miscellaneous}} \\
\hline
\hline
\iffalse ∼ & 223C & \backslash\texttt{sim} & Homotopies \\
\hline\fi
≃ & 2243 & \backslash\texttt{equiv}  & Equivalences \\
\hline
≅ & 2245 & \backslash\texttt{cong}   & Isomorphisms \\
\hline
⊥ & 22A5 & \backslash\texttt{bot}    & The overobject classifier \\
\hline
∞ & 221E & \backslash\texttt{infty}  & Infinity categories and infinity groupoids\\ 
 \hline
\end{tabular}
\end{center}}

Of these, the characters $\texttt{ꜝ,ꜞ,𛲔}$ and $\texttt{ॱ}$ do not have VSCode shortcuts, and so I provide alternatives for them. Possibly they will have to be changed if this work assimilates into a larger project.\\

It is not possible to copy the from the pdf to the clipboard while preserving the integrity of the code. To see the official Lean 4 file please click the link on the top right of the front page or \href{https://github.com/linlib/CategoryTheory/tree/main}{this}.



%LEAN: 
\begin{center}
\begin{tcolorbox}[width=5in,colback={white},title={\begin{center}\texttt{Lean \thelcounter} \addtocounter{lcounter}{1}  \end{center}},colbacktitle=Blue,coltitle=black]
\begin{minted}[breaklines, escapeinside=||]{lean}


import Mathlib.CategoryTheory.Bicategory.Basic
import Mathlib.CategoryTheory.Types 
import Mathlib.CategoryTheory.DiscreteCategory
import Mathlib.Combinatorics.Quiver.Basic
import Mathlib.CategoryTheory.Category.Init
import Aesop
import Init
import Mathlib.CategoryTheory.DiscreteCategory
import Mathlib.CategoryTheory.Bicategory.Strict
import Mathlib.CategoryTheory.ConcreteCategory.Bundled
import Mathlib.CategoryTheory.Functor.Basic
import Init.Core
import Mathlib.CategoryTheory.Category.Cat

import TheWhiteheadTheorem

-- #check 
-- #

\end{minted}
\end{tcolorbox}
\end{center}


\newpage
\begin{center}

\pagecolor{white}
\color{black}




\end{center}

\thispagestyle{empty}




\newpage
\pagecolor{white}
\color{black}
\ \\
\ \\
\thispagestyle{empty}
\begin{center}
Copyright\ \textcopyright \ June 2023 Elliot Dean Young.\ All rights reserved.\\
\end{center}
\large %%%%%%%% HERE IS THE large LARGE size textsize set text size
\newpage 
\ \\
\ \\
\ \\
\ \\
\ \\
\ \\
\ \\
\ \\
\ \\
\ \\
\ \\
\thispagestyle{empty}
 
\newpage
\section{Contents}


{
\footnotesize
\begin{longtable}{|| l || l ||} 
\hline
\multicolumn{1}{||c||}{$\texttt{Section}$} & \multicolumn{1}{|c||}{$\texttt{Description}$} \\
\hline
\hline
Unfinished & \\
\hline
Contents & \\
\hline
Unicode & \\
\hline
Introduction & \\
\hline \hline
\multicolumn{2}{||c||}{\texttt{PART I: } Internal Universes} \\
\hline \hline
\multicolumn{2}{||c||}{\texttt{Chapter 1: } ∞\_(∞-Grpd₀)} \\
\hline \hline
χ⃗𛲔 & \\
\hline 
χ⃗ॱ &  \\
\hline
D(χ⃗𛲔) &  \\
\helin
D(χ⃗ॱ) &  \\
\hline
D(∞\_(∞-Grpd₋₁)⁄-) &  \\
\hline
D([-ᵒᵖ,∞\_(∞-Grpd₋₁)]) &  \\
\hline
D(∞\_(∞-Grpd₋₁)⁄-) ≃ D([-ᵒᵖ,∞\_(∞-Grpd₋₁)]) &  \\
\hline \hline
\multicolumn{2}{||c||}{\texttt{Chapter 2: } ∞\_(∞-Grpd)} \\
\hline \hline
χ⃡𛲔 & \\
\hline
χ⃡ॱ & \\
\hline
D(χ⃡𛲔) & \\
\hline
D(χ⃡ॱ) & \\
\hline
D(∞\_(∞-Grpd)⁄-) & \\
\hline
D([-ᵒᵖ,∞\_(∞-Grpd)]) & \\
\hline
D(∞\_(∞-Grpd)⁄-) ≃ D([-ᵒᵖ,∞\_(∞-Grpd)]) & \\
\hline \hline
\multicolumn{2}{||c||}{\texttt{Chapter 3: } ∞\_(∞-Cat)} \\
\hline \hline
χ𛲔 & \\
\hline
χॱ & \\
\hline
D(χ𛲔) & \\
\hline
D(χॱ) & \\
\hline
D(∞\_(∞-Cat)⁄-) & \\
\hline
D([-ᵒᵖ,∞\_(∞-Cat)]) & \\
\hline
D(∞\_(∞-Cat)⁄-) ≃ D([-ᵒᵖ,∞\_(∞-Cat)]) & \\
\hline \hline
\multicolumn{2}{||c||}{\texttt{Chapter 4: } Monadicity, D(∞-Grpd₀), and D(∞-Grpd₀⁄X₀) ⇄ D(∞-Grpd₀⁄Y₀)} \\
\hline \hline
 & \\
\hline
 & \\
\hline \hline
\multicolumn{2}{||c||}{\texttt{Chapter 5: } Monadicity, D(∞-Grpd), and D(∞-Grpd⁄X) ⇄ D(∞-Grpd⁄Y)} \\
\hline \hline
 & \\
\hline
 & \\
\hline \hline
\multicolumn{2}{||c||}{\texttt{Chapter 6: } Monadicity, D(∞-Cat), and D(∞-Cat⁄C) ⇄ D(∞-Cat⁄D)} \\
\hline \hline
 & \\
\hline
 & \\
\hline \hline
\multicolumn{2}{||c||}{\texttt{PART II} KAN EXTENSIONS} \\
\hline \hline
\multicolumn{2}{||c||}{\texttt{Chapter 7: } Kan Extensions} \\
\hline \hline
 & \\
\hline
 & \\
\hline \hline
\multicolumn{2}{||c||}{\texttt{Chapter 8: } Kan Extensions} \\
\hline \hline
\end{longtable}
}


\iffalse 
χ⃗ : ??? ∞\_(∞-Grpd₀)/X₀ ⭢ ∞\_(∞-Grpd₀)/Y₀ 
\fi 

In this repository, I would like to think about the relationship between homotopy colimits, directed homotopy colimits, and homotopy colimits over based connected ∞-groupoids and one object ∞-groupoids, pariticularly as it concerns the six ``fibrant replace and forget" functors.\\

I would also like to incorporate two notions of the formal addition of an interval object and directed interval object, as well as six theorems concerning monadicity that are related to it.\\

\iffalse
It could make sense to define a geometric map here
\fi


\iffalse
\multicolumn{2}{||c||}{\texttt{Chapter 18: }Kan Extensions} \\
\hline
??? & ???\\
\hline
\multicolumn{2}{||c||}{\texttt{Chapter 19: }Isbell Duality} \\
\hline
??? & ???\\
\hline
\multicolumn{2}{||c||}{\texttt{Chapter 20: }Adjoints} \\
\hline
??? & ???\\
\hline
\multicolumn{2}{||c||}{\texttt{Chapter 18: }Pointed Kan Extensions} \\
\hline
??? & ???\\
\hline \hline
\fi




\begin{center}
% https://tikzcd.yichuanshen.de/#N4Igdg9gJgpgziAXAbVABwnAlgFyxMJZABgBpiBdUkANwEMAbAVxiRAB12cYAPHf4MgDCgJVxAarilAeEScA+gApJAWiF0cASgoBfEJtLpMufIRRkAjFVqMWbae3lLO3PgJXrAIgRDtu-djwEiAMzkFvTMrIggyAAi4lKyjrz8OMAKyqoaXnogGL5GgaTm1KHWEbb2Cc7JaeqabhVJwFFeFjBQAObwRKAAZgBOEAC2SGQgOBBIptQARjBgUEgBI8XhIHKA48CA9BuAKTtqAHQQUwBWAARCOll9gxPUY8PTs-OIi0VWK+uAjpB7Bydn3iCXQyeN3GiEmljCbHqAjkgEngQDrBLsABaDY4AMTUmm2IGoDDoMwYAAUDH5jCAGDBujhsSAZnMFsQ-gC7qMQQAmF4QiJQ5JyGAI5EDNEY97U2mPRaM-qA9ks5nLSFcRLQvlIlHozQiyVXUHAhYckocRWVFLw1WC9Wai5SpAy25A8EG9bbL5HY5RUUPelawFBWWIGVi+n6t5rT77V3uzRaIA
\begin{tikzcd}
{\texttt{[Cᵒᵖ,∞\_(∞-Cat)]}} \arrow[d, "(χ𛲔).obj C", bend left] \arrow[rrr, "\texttt{(e⃗.hom F)}ॱ", bend left] &  &  & {[Dᵒᵖ,∞\_\texttt{(∞-Cat)]}} \arrow[lll, "\texttt{(e⃗.hom F)}ॱ"] \arrow[d, "(χ𛲔).obj D", bend left]               \\
∞\_(∞-\texttt{Cat)⁄C} \arrow[u, "(χॱ).obj C", bend left] \arrow[rrr, "\texttt{(ω⃗.hom F)}ॱ"]                   &  &  & ∞\_(∞\texttt{-Cat)}⁄\texttt{D} \arrow[lll, "\texttt{(ω⃗.hom F)}𛲔", bend left] \arrow[u, "(χॱ).obj D", bend left]
\end{tikzcd}
\end{center}


\iffalse

 the left adjoint to precomposition in the case of the derived completion functors:

\begin{enumerate}
\item (completion of an ∞-category with respect to the directed derived category D(∞-Cat))
\begin{center}
 χ⃗  : (X : ∞-Cat) → (Y : ∞-Cat) → ∞-Cat.hom X Y → Adjunction D([-ᵒᵖ,∞-Cat]) D([-ᵒᵖ,∞-Cat] )
\end{center}
\item (completion of an ∞-groupoid with respect to the directed derived category D(∞-Grpd)) χ⃡ : [-ᵒᵖ,D(∞-Grpd)] ⇄ [-ᵒᵖ,D(∞-Grpd)]
\begin{center}
 χ⃡ : (X : ∞-Grpd) → (Y :∞-Grpd) → ∞-Grpd.hom X Y → Adjunction [-ᵒᵖ,D(∞-Grpd)] [-ᵒᵖ,D(∞-Grpd)]
\end{center}
\item (completion of a based connected ∞-groupoid with respect to the category of based connected ∞-groupoids) χ : [-ᵒᵖ,D(∞-Grpd₀)] ⇄ [-ᵒᵖ,D(∞-Grpd₀)] :
\begin{center}
 χ : (X : ∞-Grpd₀) → (Y :∞-Grpd₀) → ∞-Grpd₀.hom X Y → Adjunction [-ᵒᵖ,D(∞-Grpd₀)] [-ᵒᵖ,D(∞-Grpd₀)]
\end{center}
\end{enumerate}



\begin{enumerate}
\item E, e and the various `up to homotopy' structures (six)
\item 
\end{enumerate}




I like to use a notation of upper and lower dots for the left and right adjoint in $\texttt{Lean 4}$, but it requires specific fonts that not all systems have, and the two different characters do not always display like one another.\\

These three functors produce 

\begin{enumerate}
\item Links back to the category section concerning the category of elements and how it can be used to express pointed Kan extensions.
\item χॱ.hom : Functor D(∞-Cat) ⇄ D(∞-Cat) : χ𛲔
\item χ
\item \iffalse ∞_(∞-Lan) : D(∞-Cat) ⭢ \fi
\end{enumerate}

After establishing several interesting features of Lan D(∞-Cat),  Fᵒᵖ : [Cᵒᵖ,D(∞-Cat)] ↔ [Dᵒᵖ,D(∞-Cat)] : Cmp



[D(∞-Grpd), D(∞-Grpd₀)]

Some further goals:
\begin{enumerate}
\item Connecting the dots with various Mathlib 4 structures:
\begin{enumerate}
\item Projective space over an algebraically closed field as an ∞-topos
\item The Segre embedding as a map of ∞-topoi
\item Embedding various common categories into ∞-topoi
\item 
\end{enumerate}
\end{enumerate}

\section{Lan D(∞-Cat)}


\iffalse

[-,∞-Cat]
[-,∞-Grpd]
[-,∞-Grpd₀]


[-,∞-Cat]
[-,∞-Grpd]
[-,∞-Grpd₀]

\fi


\chapter{ETCC Signature 3}

ETCC signature 3 says that 

\begin{enumerate}
\item D(∞\_(∞-Cat)) classifies D(∞\_(∞-Cat)⁄-) : ∞\_(∞-Cat) ⭢ ∞\_(∞-Cat)
\item 
\end{enumerate}

\iffalse
\begin{definition}
The separable closure
\end{definition}

\begin{definition}
The maximal unramified extension
\end{definition}

\begin{definition}
The (abelian) separable closure
\end{definition}

\begin{definition}
The (abelian) maximal unramified extension
\end{definition} 
\fi



\newpage
{
\Huge 
\begin{center}
\ \\
\ \\
\texttt{Works Cited}
\ \\
\ \\
\end{center}
\thispagestyle{empty}
}



\begin{enumerate}
\item https://florisvandoorn.com/talks/Bonn2018spectralsequences.pdf
\item https://arxiv.org/pdf/2303.02382.pdf
\end{enumerate}

Further reading:

\begin{enumerate}
\item https://ncatlab.org/nlab/show/Chern+class
\end{enumerate}
\fi


\iffalse
\multicolumn{2}{||c||}{\texttt{Chapter 17: }Pullback Systems} \\
\hline
$\texttt{p\_(Γ) f}$ & The directed homotopy pullback adjunction \\ 
\hline
$\texttt{P\_(Γ) f} $ & The directed path-space adjunction \\ 
\hline
$\texttt{*\_(Γ)}$  & The point \\ 
\hline
$\texttt{∞\_(Γ)}$ & The universe object \\ 
\hline
$\texttt{⊥\_(Γ)}$ & The overobject classifier  \\
\hline
$\texttt{χ\_(Γ)}$ & The characteristic morphism \\
\hline \hline
\fi


\iffalse

  \chapter{Chapter 2: Pullback Systems} \\\
\\\
\section{$\texttt{pullback\_systems}$}
$\texttt{pullback\_systems}$ is a type defined in this kernel to handle derived directed homotopy pullback and directed homotopy pullback of something with itself (which we take to be directed path space [Δ1,-]). It also stores the information of the category D(∞-Cat), as well as D(∞-Cat⁄C) for each C : D(∞-Cat). D(∞-Cat) has a universe object ∞ as well, and a point *, and a map ⊥ : * → ∞. Various constructions associated to ∞-categories with a kind of universe object form the only example of a pullback system that we will use here, and yet it may prove to be useful in understanding how to force a situation in which the Whitehead theorem applies later on. \\
%LEAN: defining a pullback system
\begin{center} \begin{tcolorbox}[width=5in,colback={white},title={\begin{center}\texttt{Lean \thelcounter} \addtocounter{lcounter}{1} \end{center}},colbacktitle=Blue,coltitle=black] \begin{minted}[breaklines, escapeinside=||]{lean}

 /-
structure pullback_system where
Obj : Cat
Pnt : Obj
CmpObj : Obj.Obj → Cat CmpHom : (C : Obj.Obj) →
(D : Obj.Obj) →
(F : Obj.Hom C D) → (Functor (CmpObj C) (CmpObj D)) -- note this may need tweaking, but it should produce a functor F : D(∞-Cat/D) → D(∞-Cat/C) for each
--
CmpIdn : (C : Obj.Obj) → ((CmpHom C C (𝟙_(Obj) C)) = 𝟏_(𝐂𝐚𝐭) (CmpObj C)) CmpCmp : (C : Obj.Obj) → (D : Obj.Obj) → (E : Obj.Obj) → (F : Obj.Hom C D) → (G :
Obj.Hom D E) → (((CmpHom D E G) •_(𝐂𝐚𝐭) (CmpHom C D F)) = CmpHom C E (G ∘_(Obj) F))
Fix : Obj ≃_(𝐂𝐚𝐭) (CmpObj Pnt) Pul : (C : Obj.Obj) →
(D : Obj.Obj) →
(F : Obj.Hom C D) →
(𝐂𝐚𝐭.Hom (CmpObj D) (CmpObj C)).Obj
η : (C : Obj.Obj) → (D : Obj.Obj) → (F : Obj.Hom C D) → ((𝐂𝐚𝐭.Hom (CmpObj C) (CmpObj C)).Hom (𝟏_(𝐂𝐚𝐭) (CmpObj C)) ((Pul C D F) •_(𝐂𝐚𝐭) (CmpHom C D F)))
ε : (C : Obj.Obj) → (D : Obj.Obj) → (F : Obj.Hom C D) → ((𝐂𝐚𝐭.Hom (CmpObj D) (CmpObj D)).Hom ((CmpHom C D F) •_(𝐂𝐚𝐭) (Pul C D F)) (𝟏_(𝐂𝐚𝐭) (CmpObj D)))
Id1 : (C : Obj.Obj) → (D : Obj.Obj) → (F : Obj.Hom C D) → (AdjId1 𝐂𝐚𝐭 (CmpObj C) (CmpObj D) (CmpHom C D F) (Pul C D F) (η C D F) (ε C D F))
Id2 : (C : Obj.Obj) → (D : Obj.Obj) → (F : Obj.Hom C D) → (AdjId2 𝐂𝐚𝐭 (CmpObj C) (CmpObj D) (CmpHom C D F) (Pul C D F) (η C D F) (ε C D F))
Inf : Obj.Obj
Ovr : Obj.Hom Pnt Inf
Chi : (C : Obj.Obj) → (F : (CmpObj C).Obj) → ((Obj.Hom) C Inf)
-- ??? : (C : Obj.Obj) → (F : (Cmp C).Obj) → ((Pul C Inf (χ C f)).Hom Pnt Inf Ovr = f) -/
\end{minted} \end{tcolorbox} \end{center}
The last axiom above is a bit tricky, but it is related to straightening and unstraightening.\\ We use the notation like this:
%LEAN:

 \begin{center} \begin{tcolorbox}[width=5in,colback={white},title={\begin{center}\texttt{Lean \thelcounter} \addtocounter{lcounter}{1} \end{center}},colbacktitle=Blue,coltitle=black] \begin{minted}[breaklines, escapeinside=||]{lean}
notation "D(" Γ "⁄-)" => "Γ"
\end{minted} \end{tcolorbox} \end{center}
In the remainder of this section, I made some boxes which altogether filled out make pullback systems into a category.\\
%LEAN: defining a map of pullback systems
\begin{center} \begin{tcolorbox}[width=5in,colback={white},title={\begin{center}\texttt{Lean \thelcounter} \addtocounter{lcounter}{1} \end{center}},colbacktitle=Blue,coltitle=black] \begin{minted}[breaklines, escapeinside=||]{lean}
-- defining a map of pullback systems
structure PulHom (Γ1 : pullback_system) (Γ2 : pullback_system) where
Obj : (𝐂𝐚𝐭.Hom Γ1.Obj Γ2.Obj).Obj
Obj2 : (C : Γ1.Obj.Obj) → (𝐂𝐚𝐭.Hom (Γ1.CmpObj C) (Γ2.CmpObj (Obj.Obj C))).Obj -- plays well with CmpHom?
-- Idn
-- Cmp
\end{minted} \end{tcolorbox} \end{center}
%LEAN: defining the identity map of two pullback systems
\begin{center} \begin{tcolorbox}[width=5in,colback={white},title={\begin{center}\texttt{Lean \thelcounter} \addtocounter{lcounter}{1} \end{center}},colbacktitle=Blue,coltitle=black] \begin{minted}[breaklines, escapeinside=||]{lean}
-- defining the identity map of two pullback systems def PulIdn (X : pullback_system) : PulHom X X := sorry
\end{minted} \end{tcolorbox}

 \end{center}
%LEAN: defining the composition of two pullback systems
\begin{center} \begin{tcolorbox}[width=5in,colback={white},title={\begin{center}\texttt{Lean \thelcounter} \addtocounter{lcounter}{1} \end{center}},colbacktitle=Blue,coltitle=black] \begin{minted}[breaklines, escapeinside=||]{lean}
-- defining the composition of two pullback systems
def PulCmp (X : pullback_system) (Y : pullback_system) (Z : pullback_system) (_ : PulHom X Y) (_ : PulHom Y Z) : PulHom X Z := sorry
\end{minted} \end{tcolorbox} \end{center}
%LEAN: proving the first identity law for maps of pullback systems
\begin{center} \begin{tcolorbox}[width=5in,colback={white},title={\begin{center}\texttt{Lean \thelcounter} \addtocounter{lcounter}{1} \end{center}},colbacktitle=Blue,coltitle=black] \begin{minted}[breaklines, escapeinside=||]{lean}
-- proving the first identity law for maps of pullback systems
def PulId1 (X : pullback_system) (Y : pullback_system) (f : PulHom X Y) : PulCmp X Y Y f (PulIdn Y) = f := sorry
\end{minted} \end{tcolorbox} \end{center}
%LEAN: proving the second identity law for maps of pullback systems
\begin{center} \begin{tcolorbox}[width=5in,colback={white},title={\begin{center}\texttt{Lean \thelcounter} \addtocounter{lcounter}{1} \end{center}},colbacktitle=Blue,coltitle=black] \begin{minted}[breaklines, escapeinside=||]{lean}
-- proving the first identity law for maps of pullback systems
def PulId2 (X : pullback_system) (Y : pullback_system) (f : PulHom X Y) : PulCmp X X Y (PulIdn X) f = f := sorry
\end{minted}

 \end{tcolorbox} \end{center}
%LEAN: proving the associativity law for maps of pullback systems
\begin{center} \begin{tcolorbox}[width=5in,colback={white},title={\begin{center}\texttt{Lean \thelcounter} \addtocounter{lcounter}{1} \end{center}},colbacktitle=Blue,coltitle=black] \begin{minted}[breaklines, escapeinside=||]{lean}
def PulAss (W : pullback_system) (X : pullback_system) (Y : pullback_system) (Z : pullback_system) (f : PulHom W X) (g : PulHom X Y) (h : PulHom Y Z) : PulCmp W X Z f (PulCmp X Y Z g h) = PulCmp W Y Z (PulCmp W X Y f g) h := sorry
\end{minted} \end{tcolorbox} \end{center}
%LEAN: constructing the category Pul of pullback systems
\begin{center} \begin{tcolorbox}[width=5in,colback={white},title={\begin{center}\texttt{Lean \thelcounter} \addtocounter{lcounter}{1} \end{center}},colbacktitle=Blue,coltitle=black] \begin{minted}[breaklines, escapeinside=||]{lean}
-- constructing the category Pul of pullback systems
def Pul : category := {Obj := pullback_system, Hom := PulHom, Idn := PulIdn, Cmp := PulCmp, Id1 := PulId1, Id2 := PulId2, Ass := PulAss}
notation "Loc" => Pul
\end{minted} \end{tcolorbox} \end{center}
\section{The Pullback System of Infinity Categories}
The pullback system will be able to encapsulate information associated to ∞-Cat, D(∞-Cat),
and D(∞-Cat⁄C), namely the derived directed homotopy pullback functor and its adjoint D(ω𛲔.hom f), derived postcomposition D(ωॱ.hom f).\\
$D(Γ)$ is the ``Obj" component from the above structure: %LEAN:

 \begin{center} \begin{tcolorbox}[width=5in,colback={white},title={\begin{center}\texttt{Lean \thelcounter} \addtocounter{lcounter}{1} \end{center}},colbacktitle=Blue,coltitle=black] \begin{minted}[breaklines, escapeinside=||]{lean}
def D (Γ : pulback_system) := Γ.Obj
\end{minted} \end{tcolorbox} \end{center}
Meanwhile, the categories $D($∞$-Cat⁄C)$ are formed from the the $\texttt{CmpObj}$ components in the above like this:
%LEAN:
\begin{center} \begin{tcolorbox}[width=5in,colback={white},title={\begin{center}\texttt{Lean \thelcounter} \addtocounter{lcounter}{1} \end{center}},colbacktitle=Blue,coltitle=black] \begin{minted}[breaklines, escapeinside=||]{lean}
def derived_category (Γ : pullback_system) (C : Γ.Obj.Obj) : 𝐂𝐚𝐭.Obj := Γ.CmpObj C
\end{minted} \end{tcolorbox} \end{center}
%LEAN:
\begin{center} \begin{tcolorbox}[width=5in,colback={white},title={\begin{center}\texttt{Lean \thelcounter} \addtocounter{lcounter}{1} \end{center}},colbacktitle=Blue,coltitle=black] \begin{minted}[breaklines, escapeinside=||]{lean}
-- notation "Cmp_(" Γ ")" => derived_category Γ notation "D(" Γ "⁄" C ")" => derived_category Γ C
\end{minted} \end{tcolorbox} \end{center}
\section{$\texttt{ω⃗\_(Γ) f}$}

 Each pullback system features a construction ω⃗ which is the directed homotopy pullback in the case of ∞-categories.\\
%LEAN: assembling the adjunction ω⃗_(Γ)
\begin{center} \begin{tcolorbox}[width=5in,colback={white},title={\begin{center}\texttt{Lean \thelcounter} \addtocounter{lcounter}{1} \end{center}},colbacktitle=Blue,coltitle=black] \begin{minted}[breaklines, escapeinside=||]{lean}
def directed_homotopy_pullback (Γ : pullback_system) (E : Γ.Obj.Obj) (B : Γ.Obj.Obj) (f : Γ.Obj.Hom E B) : (adjunction 𝐂𝐚𝐭) := sorry
\end{minted} \end{tcolorbox} \end{center}
%LEAN: notation for ω⃗_(Γ)
\begin{center} \begin{tcolorbox}[width=5in,colback={white},title={\begin{center}\texttt{Lean \thelcounter} \addtocounter{lcounter}{1} \end{center}},colbacktitle=Blue,coltitle=black] \begin{minted}[breaklines, escapeinside=||]{lean}
notation : 4000 "ω⃗_(" Γ ")" => pullback Γ
\end{minted} \end{tcolorbox} \end{center}
%LEAN: (ω⃗_(Γ) 𝟙_(?) C)𛲔 = 𝟏_(?) (ω⃗_(Γ) C)𛲔
\begin{center} \begin{tcolorbox}[width=5in,colback={white},title={\begin{center}\texttt{Lean \thelcounter} \addtocounter{lcounter}{1} \end{center}},colbacktitle=Blue,coltitle=black] \begin{minted}[breaklines, escapeinside=||]{lean}
--
/-
def pIdn : -/
\end{minted} \end{tcolorbox} \end{center}

 %LEAN: (ω⃗_(Γ) f)𛲔 •_(?) (ω⃗_(Γ) g)𛲔 = (p_(Γ) f ∘_(?) g)𛲔
\begin{center} \begin{tcolorbox}[width=5in,colback={white},title={\begin{center}\texttt{Lean \thelcounter} \addtocounter{lcounter}{1} \end{center}},colbacktitle=Blue,coltitle=black] \begin{minted}[breaklines, escapeinside=||]{lean}
--
/-
def pCmp : -/
\end{minted} \end{tcolorbox} \end{center}
\section{$\texttt{*\_(Γ)}$}
The terminal object is the $\texttt{Pnt}$ component, and will be pretty easy to fill out in our
upcoming semantics:
%LEAN:
\begin{center} \begin{tcolorbox}[width=5in,colback={white},title={\begin{center}\texttt{Lean \thelcounter} \addtocounter{lcounter}{1} \end{center}},colbacktitle=Blue,coltitle=black] \begin{minted}[breaklines, escapeinside=||]{lean}
-- def terminal_object (Γ : pullback_system) : Γ.Obj.α := Γ.Pnt
\end{minted} \end{tcolorbox} \end{center}
%LEAN:
\begin{center} \begin{tcolorbox}[width=5in,colback={white},title={\begin{center}\texttt{Lean \thelcounter} \addtocounter{lcounter}{1} \end{center}},colbacktitle=Green,coltitle=black] \begin{minted}[breaklines, escapeinside=||]{lean}
notation : 3000 "*_(" Γ ")" => terminal_object Γ
\end{minted} \end{tcolorbox}

 \end{center} \section{$\texttt{∞\_(Γ)}$}
The $\texttt{Inf}$ component is the internal universe in D(∞-Cat). I didn't mention this before since it's really tricky to get right in a way which doesn't encumber the approach.\\
%LEAN:
\begin{center} \begin{tcolorbox}[width=5in,colback={white},title={\begin{center}\texttt{Lean \thelcounter} \addtocounter{lcounter}{1} \end{center}},colbacktitle=Blue,coltitle=black] \begin{minted}[breaklines, escapeinside=||]{lean}
def universe_object (Γ : pulback_system) : Γ.Obj.Obj := Γ.Inf
\end{minted} \end{tcolorbox} \end{center}
%LEAN:
\begin{center} \begin{tcolorbox}[width=5in,colback={white},title={\begin{center}\texttt{Lean \thelcounter} \addtocounter{lcounter}{1} \end{center}},colbacktitle=Blue,coltitle=black] \begin{minted}[breaklines, escapeinside=||]{lean}
notation "∞_(" Γ ")" => universe_object Γ
\end{minted} \end{tcolorbox} \end{center}
\section{$\texttt{⊥\_(Γ)}$}
Here we define the "false" map, which consists of a point mapping into the universe object. Directed homotopy pullback of F : C → ∞\_(Γ) is of very special significance, and related to Lurie's ``straightening" and ``unstraightening".\\
%LEAN: defining ⊥_(Γ) : Γ.Obj.Hom *_(Γ) ∞_(Γ)
\begin{center} \begin{tcolorbox}[width=5in,colback={white},title={\begin{center}\texttt{Lean \thelcounter} \addtocounter{lcounter}{1} \end{center}},colbacktitle=Blue,coltitle=black] \begin{minted}[breaklines, escapeinside=||]{lean}

 -- defining ⊥_(Γ) : Γ.Obj.Hom *_(Γ) ∞_(Γ)
-- def overobject_classifier (Γ : pullback_system) (p: pullback_system Γ) : (Γ.Obj.Hom *_(Γ) ∞_(Γ)) := O.Ovr
\end{minted} \end{tcolorbox} \end{center}
%LEAN:
\begin{center} \begin{tcolorbox}[width=5in,colback={white},title={\begin{center}\texttt{Lean \thelcounter} \addtocounter{lcounter}{1} \end{center}},colbacktitle=Blue,coltitle=black] \begin{minted}[breaklines, escapeinside=||]{lean}
-- notation "⊥_(" Γ ")" => overobject_classifier Γ
\end{minted} \end{tcolorbox} \end{center}
\section{$\texttt{χ\_(Γ)}$}
The $\textit{straightening}$ of \texttt{χ.obj F : D → ∞\_(∞-Cat)} an ∞-functor F : C → D is an
object in ∞-Cat ⁄ C.
%LEAN: defining χ_(Γ) : ???
\begin{center} \begin{tcolorbox}[width=5in,colback={white},title={\begin{center}\texttt{Lean \thelcounter} \addtocounter{lcounter}{1} \end{center}},colbacktitle=Blue,coltitle=black] \begin{minted}[breaklines, escapeinside=||]{lean}
-- defining χ_(Γ) : ???
-- def straightening (Γ : pullback_system) {_ : overobject_classifier Γ}
\end{minted} \end{tcolorbox} \end{center}
%LEAN:
\begin{center}

 \begin{tcolorbox}[width=5in,colback={white},title={\begin{center}\texttt{Lean \thelcounter} \addtocounter{lcounter}{1} \end{center}},colbacktitle=Blue,coltitle=black] \begin{minted}[breaklines, escapeinside=||]{lean}
-- notation "χ_(" Γ ")" =>
\end{minted} \end{tcolorbox} \end{center}
The last two components of the pullback\_system strucuture (the very last of which is currently unfilled) ensure that straightening and unstraightening hold.\\
\chapter{Set}
We next produce the simplest example of a pullback system, namely $\texttt{𝕊𝕖𝕥}$, whose object component $\texttt{𝕊𝕖𝕥.Obj}$ is the category $\texttt{Set}$ previously constructed. It is the only example of a pullback system that we construct besides the main one.\\
I haven't filled any of this out, but it is much easier to see how set maps f : X → Y correspond to set maps Y → ∞\_(𝕊𝕖𝕥) than for the case of ∞-C𝕒𝕥 (which is the mentioned pullback system from which we will obtain such objects as D(∞-Cat/C) and D(∞-Cat).\\
Currently I don't have any of this filled out, but keep in mind that the analogue of ω⃗ here is simply pullback of sets (much simpler than directed derived directed homotopy pullback).\\
\chapter{Kan Extensions}
{\footnotesize \begin{center} \begin{tabular}{|| l || l ||}
\hline
$\texttt{Section}$ & $\texttt{Description}$ \\
\hline \hline
\texttt{[Cop,Set] ⇄ [Dop,Set]} & The left Kan extension \\ \hline
\texttt{[C,Set]op ⇆ [D,Set]op} & The right Kan extension \\ \hline

 \end{tabular} \end{center}} \\\
\\\
%LEAN:
\begin{center} \begin{tcolorbox}[width=5in,colback={white},title={\begin{center}\texttt{Lean \thelcounter} \addtocounter{lcounter}{1} \end{center}},colbacktitle=Blue,coltitle=black] \begin{minted}[breaklines, escapeinside=||]{lean}
-- def el (Γ : pullback_system) (C : D(Γ).Obj) (F : D(Γ).Hom C ∞_(Γ)) := (Γ.Pul C ∞_(Γ) F).Hom *_(Γ) ∞_(Γ) ⊥_(Γ)
-- #check el /-
-/
\end{minted} \end{tcolorbox} \end{center}
\section{\texttt{[Cop,∞-Cat] ⇄ [Dop,∞-Cat]}}
%LEAN: defining the left Kan extension on objects
\begin{center} \begin{tcolorbox}[width=5in,colback={white},title={\begin{center}\texttt{Lean \thelcounter} \addtocounter{lcounter}{1} \end{center}},colbacktitle=Blue,coltitle=black] \begin{minted}[breaklines, escapeinside=||]{lean}
-- (Lan F).α /-
(Lan F).α -/
\end{minted} \end{tcolorbox} \end{center}

 %LEAN:
\begin{center} \begin{tcolorbox}[width=5in,colback={white},title={\begin{center}\texttt{Lean \thelcounter} \addtocounter{lcounter}{1} \end{center}},colbacktitle=Blue,coltitle=black] \begin{minted}[breaklines, escapeinside=||]{lean}
-- (Lan F).Hom /-
-/
\end{minted} \end{tcolorbox} \end{center}
%LEAN:
\begin{center} \begin{tcolorbox}[width=5in,colback={white},title={\begin{center}\texttt{Lean \thelcounter} \addtocounter{lcounter}{1} \end{center}},colbacktitle=Blue,coltitle=black] \begin{minted}[breaklines, escapeinside=||]{lean}
-- (Lan F).Idn /-
-/
\end{minted} \end{tcolorbox} \end{center}
%LEAN:
\begin{center} \begin{tcolorbox}[width=5in,colback={white},title={\begin{center}\texttt{Lean \thelcounter} \addtocounter{lcounter}{1} \end{center}},colbacktitle=Blue,coltitle=black] \begin{minted}[breaklines, escapeinside=||]{lean}
-- (Lan F).Cmp /-
-/

 \end{minted} \end{tcolorbox} \end{center}
%LEAN:
\begin{center} \begin{tcolorbox}[width=5in,colback={white},title={\begin{center}\texttt{Lean \thelcounter} \addtocounter{lcounter}{1} \end{center}},colbacktitle=Blue,coltitle=black] \begin{minted}[breaklines, escapeinside=||]{lean}
-- Lan F /-
-/
\end{minted} \end{tcolorbox} \end{center}
%LEAN: unit of the left Kan extension on objects
\begin{center} \begin{tcolorbox}[width=5in,colback={white},title={\begin{center}\texttt{Lean \thelcounter} \addtocounter{lcounter}{1} \end{center}},colbacktitle=Blue,coltitle=black] \begin{minted}[breaklines, escapeinside=||]{lean}
-- unit of the left Kan extension on objects /-
-/
\end{minted} \end{tcolorbox} \end{center}
%LEAN: unit of the left Kan extension naturality
\begin{center} \begin{tcolorbox}[width=5in,colback={white},title={\begin{center}\texttt{Lean \thelcounter} \addtocounter{lcounter}{1} \end{center}},colbacktitle=Blue,coltitle=black] \begin{minted}[breaklines, escapeinside=||]{lean}
-- unit of the left Kan extension naturality

 /- -/
\end{minted} \end{tcolorbox} \end{center}
%LEAN: unit of the left Kan extension
\begin{center} \begin{tcolorbox}[width=5in,colback={white},title={\begin{center}\texttt{Lean \thelcounter} \addtocounter{lcounter}{1} \end{center}},colbacktitle=Blue,coltitle=black] \begin{minted}[breaklines, escapeinside=||]{lean}
-- unit of the left Kan extension /-
-/
\end{minted} \end{tcolorbox} \end{center}
%LEAN: counit of the left Kan extension on objects
\begin{center} \begin{tcolorbox}[width=5in,colback={white},title={\begin{center}\texttt{Lean \thelcounter} \addtocounter{lcounter}{1} \end{center}},colbacktitle=Blue,coltitle=black] \begin{minted}[breaklines, escapeinside=||]{lean}
-- counit of the left Kan extension on objects /-
-/
\end{minted} \end{tcolorbox} \end{center}
%LEAN: counit of the left Kan extension naturality
\begin{center}

 \begin{tcolorbox}[width=5in,colback={white},title={\begin{center}\texttt{Lean \thelcounter} \addtocounter{lcounter}{1} \end{center}},colbacktitle=Blue,coltitle=black] \begin{minted}[breaklines, escapeinside=||]{lean}
-- counit of the left Kan extension naturality /-
-/
\end{minted} \end{tcolorbox} \end{center}
%LEAN: counit of the left Kan extension
\begin{center} \begin{tcolorbox}[width=5in,colback={white},title={\begin{center}\texttt{Lean \thelcounter} \addtocounter{lcounter}{1} \end{center}},colbacktitle=Blue,coltitle=black] \begin{minted}[breaklines, escapeinside=||]{lean}
-- counit of the left Kan extension /-
-/
\end{minted} \end{tcolorbox} \end{center}
%LEAN: first triangle identity of the left Kan extension
\begin{center} \begin{tcolorbox}[width=5in,colback={white},title={\begin{center}\texttt{Lean \thelcounter} \addtocounter{lcounter}{1} \end{center}},colbacktitle=Blue,coltitle=black] \begin{minted}[breaklines, escapeinside=||]{lean}
-- first triangle identity of the left Kan extension /-
-/
\end{minted} \end{tcolorbox} \end{center}

 %LEAN: second triangle identity of the left Kan extension
\begin{center} \begin{tcolorbox}[width=5in,colback={white},title={\begin{center}\texttt{Lean \thelcounter} \addtocounter{lcounter}{1} \end{center}},colbacktitle=Blue,coltitle=black] \begin{minted}[breaklines, escapeinside=||]{lean}
-- second triangle identity of the left Kan extension /-
-/
\end{minted} \end{tcolorbox} \end{center}
%LEAN: assembling the left Kan extension adjunction
\begin{center} \begin{tcolorbox}[width=5in,colback={white},title={\begin{center}\texttt{Lean \thelcounter} \addtocounter{lcounter}{1} \end{center}},colbacktitle=Blue,coltitle=black] \begin{minted}[breaklines, escapeinside=||]{lean}
-- assembling the left Kan extension adjunction /-
-/
\end{minted} \end{tcolorbox} \end{center}
\section{\texttt{[C,∞-Cat]op ⇆ [D,∞-Cat]op}}
%LEAN: defining the left Kan extension on objects
\begin{center} \begin{tcolorbox}[width=5in,colback={white},title={\begin{center}\texttt{Lean \thelcounter} \addtocounter{lcounter}{1} \end{center}},colbacktitle=Blue,coltitle=black] \begin{minted}[breaklines, escapeinside=||]{lean}
-- constructing Ran C Φ F on objects /-
-/

 \end{minted} \end{tcolorbox} \end{center}
%LEAN:
\begin{center} \begin{tcolorbox}[width=5in,colback={white},title={\begin{center}\texttt{Lean \thelcounter} \addtocounter{lcounter}{1} \end{center}},colbacktitle=Blue,coltitle=black] \begin{minted}[breaklines, escapeinside=||]{lean}
-- constructing Ran C Φ F on morphisms /-
-/
\end{minted} \end{tcolorbox} \end{center}
%LEAN: proving the identity law of Ran C Φ F
\begin{center} \begin{tcolorbox}[width=5in,colback={white},title={\begin{center}\texttt{Lean \thelcounter} \addtocounter{lcounter}{1} \end{center}},colbacktitle=Blue,coltitle=black] \begin{minted}[breaklines, escapeinside=||]{lean}
-- proving the identity law of Ran C Φ F /-
-/
\end{minted} \end{tcolorbox} \end{center}
%LEAN: proving compositionality of the right adjoint in the right Kan extension
\begin{center} \begin{tcolorbox}[width=5in,colback={white},title={\begin{center}\texttt{Lean \thelcounter} \addtocounter{lcounter}{1} \end{center}},colbacktitle=Blue,coltitle=black] \begin{minted}[breaklines, escapeinside=||]{lean}

 -- proving compositionality of the right adjoint in the right Kan extension /-
-/
\end{minted} \end{tcolorbox} \end{center}
%LEAN: assembling the right adjoint of the right Kan extension
\begin{center} \begin{tcolorbox}[width=5in,colback={white},title={\begin{center}\texttt{Lean \thelcounter} \addtocounter{lcounter}{1} \end{center}},colbacktitle=Blue,coltitle=black] \begin{minted}[breaklines, escapeinside=||]{lean}
-- assembling the right adjoint of the right Kan extension /-
-/
\end{minted} \end{tcolorbox} \end{center}
%LEAN: unit of the right Kan extension on objects
\begin{center} \begin{tcolorbox}[width=5in,colback={white},title={\begin{center}\texttt{Lean \thelcounter} \addtocounter{lcounter}{1} \end{center}},colbacktitle=Blue,coltitle=black] \begin{minted}[breaklines, escapeinside=||]{lean}
-- unit of the right Kan extension on objects /-
-/
\end{minted} \end{tcolorbox} \end{center}
%LEAN: unit of the right Kan extension naturality
\begin{center}

 \begin{tcolorbox}[width=5in,colback={white},title={\begin{center}\texttt{Lean \thelcounter} \addtocounter{lcounter}{1} \end{center}},colbacktitle=Blue,coltitle=black] \begin{minted}[breaklines, escapeinside=||]{lean}
-- unit of the right Kan extension naturality /-
-/
\end{minted} \end{tcolorbox} \end{center}
%LEAN: unit of the right Kan extension
\begin{center} \begin{tcolorbox}[width=5in,colback={white},title={\begin{center}\texttt{Lean \thelcounter} \addtocounter{lcounter}{1} \end{center}},colbacktitle=Blue,coltitle=black] \begin{minted}[breaklines, escapeinside=||]{lean}
-- unit of the right Kan extension /-
-/
\end{minted} \end{tcolorbox} \end{center}
%LEAN: counit of the right Kan extension on objects
\begin{center} \begin{tcolorbox}[width=5in,colback={white},title={\begin{center}\texttt{Lean \thelcounter} \addtocounter{lcounter}{1} \end{center}},colbacktitle=Blue,coltitle=black] \begin{minted}[breaklines, escapeinside=||]{lean}
-- counit of the right Kan extension on objects /-
-/
\end{minted} \end{tcolorbox} \end{center}

 %LEAN: counit of the right Kan extension naturality
\begin{center} \begin{tcolorbox}[width=5in,colback={white},title={\begin{center}\texttt{Lean \thelcounter} \addtocounter{lcounter}{1} \end{center}},colbacktitle=Blue,coltitle=black] \begin{minted}[breaklines, escapeinside=||]{lean}
-- counit of the right Kan extension naturality /-
-/
\end{minted} \end{tcolorbox} \end{center}
%LEAN: counit of the right Kan extension
\begin{center} \begin{tcolorbox}[width=5in,colback={white},title={\begin{center}\texttt{Lean \thelcounter} \addtocounter{lcounter}{1} \end{center}},colbacktitle=Blue,coltitle=black] \begin{minted}[breaklines, escapeinside=||]{lean}
-- counit of the right Kan extension /-
-/
\end{minted} \end{tcolorbox} \end{center}
%LEAN: first triangle identity of the right Kan extension
\begin{center} \begin{tcolorbox}[width=5in,colback={white},title={\begin{center}\texttt{Lean \thelcounter} \addtocounter{lcounter}{1} \end{center}},colbacktitle=Blue,coltitle=black] \begin{minted}[breaklines, escapeinside=||]{lean}
-- first triangle identity of the right Kan extension /-
-/ \end{minted}

 \end{tcolorbox} \end{center}
%LEAN: second triangle identity of the right Kan extension
\begin{center} \begin{tcolorbox}[width=5in,colback={white},title={\begin{center}\texttt{Lean \thelcounter} \addtocounter{lcounter}{1} \end{center}},colbacktitle=Blue,coltitle=black] \begin{minted}[breaklines, escapeinside=||]{lean}
-- second triangle identity of the right Kan extension /-
-/
\end{minted} \end{tcolorbox} \end{center}
%LEAN: assembling the right Kan extension adjunction
\begin{center} \begin{tcolorbox}[width=5in,colback={white},title={\begin{center}\texttt{Lean \thelcounter} \addtocounter{lcounter}{1} \end{center}},colbacktitle=Blue,coltitle=black] \begin{minted}[breaklines, escapeinside=||]{lean}
-- assembling the right Kan extension adjunction /-
-/
\end{minted} \end{tcolorbox} \end{center}
\chapter{Isbell Duality}
Isbell duality is perhaps the latest development in category theory which we include. It emerged from efforts to understand the \href{https://ncatlab.org/nlab/show/ duality+between+algebra+and+geometry}{duality between Geometry and Algebra}, but is not so widely recognized as the cannon established from Mac Lane's seminal text.
\section{\texttt{[Cop,Set] ⇄ [C,Set]op}}
  
 %LEAN: defining the Isbell adjunction on objects
\begin{center} \begin{tcolorbox}[width=5in,colback={white},title={\begin{center}\texttt{Lean \thelcounter} \addtocounter{lcounter}{1} \end{center}},colbacktitle=Blue,coltitle=black] \begin{minted}[breaklines, escapeinside=||]{lean}
-- defining the Isbell adjunction on objects /-
-- Obj
-- def LanObj
-/
\end{minted} \end{tcolorbox} \end{center}
%LEAN:
\begin{center} \begin{tcolorbox}[width=5in,colback={white},title={\begin{center}\texttt{Lean \thelcounter} \addtocounter{lcounter}{1} \end{center}},colbacktitle=Blue,coltitle=black] \begin{minted}[breaklines, escapeinside=||]{lean}
-- defining the Isbell adjunction on morphisms /-
-/
\end{minted} \end{tcolorbox} \end{center}
%LEAN: proving the identity law for the Isbell adjunction
\begin{center} \begin{tcolorbox}[width=5in,colback={white},title={\begin{center}\texttt{Lean \thelcounter} \addtocounter{lcounter}{1} \end{center}},colbacktitle=Blue,coltitle=black] \begin{minted}[breaklines, escapeinside=||]{lean}
-- proving the identity law for the Isbell adjunction /-

 -/
\end{minted} \end{tcolorbox} \end{center}
%LEAN: proving compositionality for the Isbell adjunction
\begin{center} \begin{tcolorbox}[width=5in,colback={white},title={\begin{center}\texttt{Lean \thelcounter} \addtocounter{lcounter}{1} \end{center}},colbacktitle=Blue,coltitle=black] \begin{minted}[breaklines, escapeinside=||]{lean}
-- proving compositionality for the Isbell adjunction /-
-/
\end{minted} \end{tcolorbox} \end{center}
%LEAN: assembling the functor of the Isbell adjunction
\begin{center} \begin{tcolorbox}[width=5in,colback={white},title={\begin{center}\texttt{Lean \thelcounter} \addtocounter{lcounter}{1} \end{center}},colbacktitle=Blue,coltitle=black] \begin{minted}[breaklines, escapeinside=||]{lean}
-- assembling the functor of the Isbell adjunction /-
-/
\end{minted} \end{tcolorbox} \end{center}
%LEAN: unit of the Isbell adjunction on objects
\begin{center} \begin{tcolorbox}[width=5in,colback={white},title={\begin{center}\texttt{Lean \thelcounter} \addtocounter{lcounter}{1} \end{center}},colbacktitle=Blue,coltitle=black] \begin{minted}[breaklines, escapeinside=||]{lean}

 -- unit of the Isbell adjunction on objects /-
-/
\end{minted} \end{tcolorbox} \end{center}
%LEAN: unit of the Isbell adjunction naturality
\begin{center} \begin{tcolorbox}[width=5in,colback={white},title={\begin{center}\texttt{Lean \thelcounter} \addtocounter{lcounter}{1} \end{center}},colbacktitle=Blue,coltitle=black] \begin{minted}[breaklines, escapeinside=||]{lean}
-- unit of the Isbell adjunction naturality /-
-/
\end{minted} \end{tcolorbox} \end{center}
%LEAN: unit of the Isbell adjunction
\begin{center} \begin{tcolorbox}[width=5in,colback={white},title={\begin{center}\texttt{Lean \thelcounter} \addtocounter{lcounter}{1} \end{center}},colbacktitle=Blue,coltitle=black] \begin{minted}[breaklines, escapeinside=||]{lean}
-- unit of the Isbell adjunction /-
-/
\end{minted} \end{tcolorbox} \end{center}
%LEAN: triangle identity of the Isbell adjunction
\begin{center}

 \begin{tcolorbox}[width=5in,colback={white},title={\begin{center}\texttt{Lean \thelcounter} \addtocounter{lcounter}{1} \end{center}},colbacktitle=Blue,coltitle=black] \begin{minted}[breaklines, escapeinside=||]{lean}
-- triangle identity of the Isbell adjunction /-
-/
\end{minted} \end{tcolorbox} \end{center}
%LEAN: assembling the Isbell adjunction
\begin{center} \begin{tcolorbox}[width=5in,colback={white},title={\begin{center}\texttt{Lean \thelcounter} \addtocounter{lcounter}{1} \end{center}},colbacktitle=Blue,coltitle=black] \begin{minted}[breaklines, escapeinside=||]{lean}
-- assembling the Isbell adjunction /-
-/
\end{minted} \end{tcolorbox} \end{center}
\chapter{The Adjoint Functor Theorem}
{\footnotesize \begin{center} \begin{tabular}{|| l || l ||}
\hline
$\texttt{Section}$ & $\texttt{Description}$ \\
\hline
\hline
Right adjoints preserve limits & \\
\hline
Left adjoints preserve colimits & \\
\hline
The adjoint functor theorem for right adjoints & A theorem demonstrating the existence of a left adjoint \\

 \hline
The adjoint functor theorem for right adjoints & A theorem demonstrating the existence of a right adjoint \\
\hline
\end{tabular}
\end{center}} \\\
\\\
The adjoint functor theorem, also known as the adjoint functor existence theorem, was proved independently by three mathematicians: Daniel Kan, Saunders Mac Lane, and Samuel Eilenberg. The contribution of Kan came from his \href{https://www.ams.org/ journals/tran/1958-087-02/S0002-9947-1958-0131451-0/S0002-9947-1958-0131451-0.pdf } {mentioned paper} on adjoint functors, though it was not established in its full generality. This much appeared later in ``Categories for the Working Mathematician".\\
\section{Right adjoints preserve limits}
%LEAN: left adjoints preserve limits
\begin{center} \begin{tcolorbox}[width=5in,colback={white},title={\begin{center}\texttt{Lean \thelcounter} \addtocounter{lcounter}{1} \end{center}},colbacktitle=Blue,coltitle=black] \begin{minted}[breaklines, escapeinside=||]{lean}
-- left adjoints preserve limits /-
-/
\end{minted} \end{tcolorbox} \end{center}
\section{Left adjoints preserve colimits}
%LEAN: left adjoints preserve colimits
\begin{center}
  
 \begin{tcolorbox}[width=5in,colback={white},title={\begin{center}\texttt{Lean \thelcounter} \addtocounter{lcounter}{1} \end{center}},colbacktitle=Blue,coltitle=black] \begin{minted}[breaklines, escapeinside=||]{lean}
-- left adjoints preserve colimits /-
-/
\end{minted} \end{tcolorbox} \end{center}
\section{The adjoint functor theorem for right adjoints}
%LEAN: definition of limit preservation
\begin{center} \begin{tcolorbox}[width=5in,colback={white},title={\begin{center}\texttt{Lean \thelcounter} \addtocounter{lcounter}{1} \end{center}},colbacktitle=Blue,coltitle=black] \begin{minted}[breaklines, escapeinside=||]{lean}
-- definition of limit preservation /-
-/
\end{minted} \end{tcolorbox} \end{center}
%LEAN: limit preserving functors are right adjoint
\begin{center} \begin{tcolorbox}[width=5in,colback={white},title={\begin{center}\texttt{Lean \thelcounter} \addtocounter{lcounter}{1} \end{center}},colbacktitle=Blue,coltitle=black] \begin{minted}[breaklines, escapeinside=||]{lean}
-- limit preserving functors are right adjoint /-
this should show that the map [Cop,Set] ⇄ [Dop,Set]

 -/
\end{minted} \end{tcolorbox} \end{center}
\section{The adjoint functor theorem for left adjoints}
%LEAN: definition of colimit preservation
\begin{center} \begin{tcolorbox}[width=5in,colback={white},title={\begin{center}\texttt{Lean \thelcounter} \addtocounter{lcounter}{1} \end{center}},colbacktitle=Blue,coltitle=black] \begin{minted}[breaklines, escapeinside=||]{lean}
-- definition of colimit preservation /-
-/
\end{minted} \end{tcolorbox} \end{center}
%LEAN: colimit preserving functors are left adjoint
\begin{center} \begin{tcolorbox}[width=5in,colback={white},title={\begin{center}\texttt{Lean \thelcounter} \addtocounter{lcounter}{1} \end{center}},colbacktitle=Blue,coltitle=black] \begin{minted}[breaklines, escapeinside=||]{lean}
-- colimit preserving functors are left adjoint /-
-/
\end{minted} \end{tcolorbox} \end{center}
\chapter{Colimits}
\iffalse
-- hocolim_G * is naturally the domain of a space over BG. -- we can define a map [B,∞] → [E,∞] which:

 -- on objects sends F to f • (pullback with false of F)
-- on morphisms sends η to f • (pullback with false of η)
- [Xop,∞] formally adds in ∞-colimits - homotopy Kan extensions too
\fi
\chapter{Pointed Kan Extensions}
\begin{enumerate}
\item Expressing a pointed Kan extension as a \end{enumerate}
\iffalse
\chapter{E and B}
\begin{enumerate}
\item For $\texttt{C : Cat\_(D(Γ))}$, $\texttt{EC}$ is $\texttt{C.Obj ×\_(BC) BC}$? \item $\texttt{E}$ is ∞$\texttt{-colim}$ of the terminal object $\texttt{D(}$∞$\texttt{- Cat⁄C.Mor)⇄D(}$∞$\texttt{-Cat⁄BC)}$.
\end{enumerate}
\fi
\chapter{Kan Extensions}
{\footnotesize \begin{center} \begin{tabular}{|| l || l ||}
\hline
$\texttt{Section}$ & $\texttt{Description}$ \\
\hline \hline
\texttt{[Cop,Set] ⇄ [Dop,Set]} & The left Kan extension \\ \hline
\texttt{[C,Set]op ⇆ [D,Set]op} & The right Kan extension \\ \hline

 \end{tabular} \end{center}} \\\
\\\
%LEAN:
\begin{center} \begin{tcolorbox}[width=5in,colback={white},title={\begin{center}\texttt{Lean \thelcounter} \addtocounter{lcounter}{1} \end{center}},colbacktitle=Blue,coltitle=black] \begin{minted}[breaklines, escapeinside=||]{lean}
-- def el (Γ : pullback_system) (C : D(Γ).Obj) (F : D(Γ).Hom C ∞_(Γ)) := (Γ.Pul C ∞_(Γ) F).Hom *_(Γ) ∞_(Γ) ⊥_(Γ)
-- #check el
/- -/
\end{minted} \end{tcolorbox} \end{center}
\section{\texttt{[Cop,∞-Cat] ⇄ [Dop,∞-Cat]}}
%LEAN: defining the left Kan extension on objects
\begin{center} \begin{tcolorbox}[width=5in,colback={white},title={\begin{center}\texttt{Lean \thelcounter} \addtocounter{lcounter}{1} \end{center}},colbacktitle=Blue,coltitle=black] \begin{minted}[breaklines, escapeinside=||]{lean}
-- (Lan F).Obj /-
(Lan F).Obj -/
\end{minted} \end{tcolorbox} \end{center}

 %LEAN:
\begin{center} \begin{tcolorbox}[width=5in,colback={white},title={\begin{center}\texttt{Lean \thelcounter} \addtocounter{lcounter}{1} \end{center}},colbacktitle=Blue,coltitle=black] \begin{minted}[breaklines, escapeinside=||]{lean}
-- (Lan F).Hom /-
-/
\end{minted} \end{tcolorbox} \end{center}
%LEAN:
\begin{center} \begin{tcolorbox}[width=5in,colback={white},title={\begin{center}\texttt{Lean \thelcounter} \addtocounter{lcounter}{1} \end{center}},colbacktitle=Blue,coltitle=black] \begin{minted}[breaklines, escapeinside=||]{lean}
-- (Lan F).Idn /-
-/
\end{minted} \end{tcolorbox} \end{center}
%LEAN:
\begin{center} \begin{tcolorbox}[width=5in,colback={white},title={\begin{center}\texttt{Lean \thelcounter} \addtocounter{lcounter}{1} \end{center}},colbacktitle=Blue,coltitle=black] \begin{minted}[breaklines, escapeinside=||]{lean}
-- (Lan F).Cmp /-
-/

 \end{minted} \end{tcolorbox} \end{center}
%LEAN:
\begin{center} \begin{tcolorbox}[width=5in,colback={white},title={\begin{center}\texttt{Lean \thelcounter} \addtocounter{lcounter}{1} \end{center}},colbacktitle=Blue,coltitle=black] \begin{minted}[breaklines, escapeinside=||]{lean}
-- Lan F /-
-/
\end{minted} \end{tcolorbox} \end{center}
%LEAN: unit of the left Kan extension on objects
\begin{center} \begin{tcolorbox}[width=5in,colback={white},title={\begin{center}\texttt{Lean \thelcounter} \addtocounter{lcounter}{1} \end{center}},colbacktitle=Blue,coltitle=black] \begin{minted}[breaklines, escapeinside=||]{lean}
-- unit of the left Kan extension on objects /-
-/
\end{minted} \end{tcolorbox} \end{center}
%LEAN: unit of the left Kan extension naturality
\begin{center} \begin{tcolorbox}[width=5in,colback={white},title={\begin{center}\texttt{Lean \thelcounter} \addtocounter{lcounter}{1} \end{center}},colbacktitle=Blue,coltitle=black] \begin{minted}[breaklines, escapeinside=||]{lean}

 -- unit of the left Kan extension naturality /-
-/
\end{minted} \end{tcolorbox} \end{center}
%LEAN: unit of the left Kan extension
\begin{center} \begin{tcolorbox}[width=5in,colback={white},title={\begin{center}\texttt{Lean \thelcounter} \addtocounter{lcounter}{1} \end{center}},colbacktitle=Blue,coltitle=black] \begin{minted}[breaklines, escapeinside=||]{lean}
-- unit of the left Kan extension /-
-/
\end{minted} \end{tcolorbox} \end{center}
%LEAN: counit of the left Kan extension on objects
\begin{center} \begin{tcolorbox}[width=5in,colback={white},title={\begin{center}\texttt{Lean \thelcounter} \addtocounter{lcounter}{1} \end{center}},colbacktitle=Blue,coltitle=black] \begin{minted}[breaklines, escapeinside=||]{lean}
-- counit of the left Kan extension on objects /-
-/
\end{minted} \end{tcolorbox} \end{center}
%LEAN: counit of the left Kan extension naturality
\begin{center}

 \begin{tcolorbox}[width=5in,colback={white},title={\begin{center}\texttt{Lean \thelcounter} \addtocounter{lcounter}{1} \end{center}},colbacktitle=Blue,coltitle=black] \begin{minted}[breaklines, escapeinside=||]{lean}
-- counit of the left Kan extension naturality /-
-/
\end{minted} \end{tcolorbox} \end{center}
%LEAN: counit of the left Kan extension
\begin{center} \begin{tcolorbox}[width=5in,colback={white},title={\begin{center}\texttt{Lean \thelcounter} \addtocounter{lcounter}{1} \end{center}},colbacktitle=Blue,coltitle=black] \begin{minted}[breaklines, escapeinside=||]{lean}
-- counit of the left Kan extension /-
-/
\end{minted} \end{tcolorbox} \end{center}
%LEAN: first triangle identity of the left Kan extension
\begin{center} \begin{tcolorbox}[width=5in,colback={white},title={\begin{center}\texttt{Lean \thelcounter} \addtocounter{lcounter}{1} \end{center}},colbacktitle=Blue,coltitle=black] \begin{minted}[breaklines, escapeinside=||]{lean}
-- first triangle identity of the left Kan extension /-
-/
\end{minted} \end{tcolorbox} \end{center}

 %LEAN: second triangle identity of the left Kan extension
\begin{center} \begin{tcolorbox}[width=5in,colback={white},title={\begin{center}\texttt{Lean \thelcounter} \addtocounter{lcounter}{1} \end{center}},colbacktitle=Blue,coltitle=black] \begin{minted}[breaklines, escapeinside=||]{lean}
-- second triangle identity of the left Kan extension /-
-/
\end{minted} \end{tcolorbox} \end{center}
%LEAN: assembling the left Kan extension adjunction
\begin{center} \begin{tcolorbox}[width=5in,colback={white},title={\begin{center}\texttt{Lean \thelcounter} \addtocounter{lcounter}{1} \end{center}},colbacktitle=Blue,coltitle=black] \begin{minted}[breaklines, escapeinside=||]{lean}
-- assembling the left Kan extension adjunction /-
-/
\end{minted} \end{tcolorbox} \end{center}
\section{\texttt{[C,∞-Cat]op ⇆ [D,∞-Cat]op}}
%LEAN: defining the left Kan extension on objects
\begin{center} \begin{tcolorbox}[width=5in,colback={white},title={\begin{center}\texttt{Lean \thelcounter} \addtocounter{lcounter}{1} \end{center}},colbacktitle=Blue,coltitle=black] \begin{minted}[breaklines, escapeinside=||]{lean}
-- constructing Ran C Φ F on objects /-
-/
\end{minted}

 \end{tcolorbox} \end{center}
%LEAN:
\begin{center} \begin{tcolorbox}[width=5in,colback={white},title={\begin{center}\texttt{Lean \thelcounter} \addtocounter{lcounter}{1} \end{center}},colbacktitle=Blue,coltitle=black] \begin{minted}[breaklines, escapeinside=||]{lean}
-- constructing Ran C Φ F on morphisms /-
-/
\end{minted} \end{tcolorbox} \end{center}
%LEAN: proving the identity law of Ran C Φ F
\begin{center} \begin{tcolorbox}[width=5in,colback={white},title={\begin{center}\texttt{Lean \thelcounter} \addtocounter{lcounter}{1} \end{center}},colbacktitle=Blue,coltitle=black] \begin{minted}[breaklines, escapeinside=||]{lean}
-- proving the identity law of Ran C Φ F /-
-/
\end{minted} \end{tcolorbox} \end{center}
%LEAN: proving compositionality of the right adjoint in the right Kan extension
\begin{center} \begin{tcolorbox}[width=5in,colback={white},title={\begin{center}\texttt{Lean \thelcounter} \addtocounter{lcounter}{1} \end{center}},colbacktitle=Blue,coltitle=black] \begin{minted}[breaklines, escapeinside=||]{lean}
-- proving compositionality of the right adjoint in the right Kan extension

 /- -/
\end{minted} \end{tcolorbox} \end{center}
%LEAN: assembling the right adjoint of the right Kan extension
\begin{center} \begin{tcolorbox}[width=5in,colback={white},title={\begin{center}\texttt{Lean \thelcounter} \addtocounter{lcounter}{1} \end{center}},colbacktitle=Blue,coltitle=black] \begin{minted}[breaklines, escapeinside=||]{lean}
-- assembling the right adjoint of the right Kan extension /-
-/
\end{minted} \end{tcolorbox} \end{center}
%LEAN: unit of the right Kan extension on objects
\begin{center} \begin{tcolorbox}[width=5in,colback={white},title={\begin{center}\texttt{Lean \thelcounter} \addtocounter{lcounter}{1} \end{center}},colbacktitle=Blue,coltitle=black] \begin{minted}[breaklines, escapeinside=||]{lean}
-- unit of the right Kan extension on objects /-
-/
\end{minted} \end{tcolorbox} \end{center}
%LEAN: unit of the right Kan extension naturality
\begin{center} \begin{tcolorbox}[width=5in,colback={white},title={\begin{center}\texttt{Lean \thelcounter} \addtocounter{lcounter}{1} \end{center}},colbacktitle=Blue,coltitle=black]

 \begin{minted}[breaklines, escapeinside=||]{lean} -- unit of the right Kan extension naturality
/- -/
\end{minted} \end{tcolorbox} \end{center}
%LEAN: unit of the right Kan extension
\begin{center} \begin{tcolorbox}[width=5in,colback={white},title={\begin{center}\texttt{Lean \thelcounter} \addtocounter{lcounter}{1} \end{center}},colbacktitle=Blue,coltitle=black] \begin{minted}[breaklines, escapeinside=||]{lean}
-- unit of the right Kan extension /-
-/
\end{minted} \end{tcolorbox} \end{center}
%LEAN: counit of the right Kan extension on objects
\begin{center} \begin{tcolorbox}[width=5in,colback={white},title={\begin{center}\texttt{Lean \thelcounter} \addtocounter{lcounter}{1} \end{center}},colbacktitle=Blue,coltitle=black] \begin{minted}[breaklines, escapeinside=||]{lean}
-- counit of the right Kan extension on objects /-
-/
\end{minted} \end{tcolorbox} \end{center}
%LEAN: counit of the right Kan extension naturality

 \begin{center} \begin{tcolorbox}[width=5in,colback={white},title={\begin{center}\texttt{Lean \thelcounter} \addtocounter{lcounter}{1} \end{center}},colbacktitle=Blue,coltitle=black] \begin{minted}[breaklines, escapeinside=||]{lean}
-- counit of the right Kan extension naturality /-
-/
\end{minted} \end{tcolorbox} \end{center}
%LEAN: counit of the right Kan extension
\begin{center} \begin{tcolorbox}[width=5in,colback={white},title={\begin{center}\texttt{Lean \thelcounter} \addtocounter{lcounter}{1} \end{center}},colbacktitle=Blue,coltitle=black] \begin{minted}[breaklines, escapeinside=||]{lean}
-- counit of the right Kan extension /-
-/
\end{minted} \end{tcolorbox} \end{center}
%LEAN: first triangle identity of the right Kan extension
\begin{center} \begin{tcolorbox}[width=5in,colback={white},title={\begin{center}\texttt{Lean \thelcounter} \addtocounter{lcounter}{1} \end{center}},colbacktitle=Blue,coltitle=black] \begin{minted}[breaklines, escapeinside=||]{lean}
-- first triangle identity of the right Kan extension /-
-/
\end{minted} \end{tcolorbox} \end{center}

 %LEAN: second triangle identity of the right Kan extension
\begin{center} \begin{tcolorbox}[width=5in,colback={white},title={\begin{center}\texttt{Lean \thelcounter} \addtocounter{lcounter}{1} \end{center}},colbacktitle=Blue,coltitle=black] \begin{minted}[breaklines, escapeinside=||]{lean}
-- second triangle identity of the right Kan extension /-
-/
\end{minted} \end{tcolorbox} \end{center}
%LEAN: assembling the right Kan extension adjunction
\begin{center} \begin{tcolorbox}[width=5in,colback={white},title={\begin{center}\texttt{Lean \thelcounter} \addtocounter{lcounter}{1} \end{center}},colbacktitle=Blue,coltitle=black] \begin{minted}[breaklines, escapeinside=||]{lean}
-- assembling the right Kan extension adjunction /-
-/
\end{minted} \end{tcolorbox} \end{center}
\chapter{Isbell Duality}
Isbell duality is perhaps the latest development in category theory which we include. It emerged from efforts to understand the \href{https://ncatlab.org/nlab/show/ duality+between+algebra+and+geometry}{duality between Geometry and Algebra}, but is not so widely recognized as the cannon established from Mac Lane's seminal text.
\section{\texttt{[Cop,Set] ⇄ [C,Set]op}}
  
 %LEAN: defining the Isbell adjunction on objects
\begin{center} \begin{tcolorbox}[width=5in,colback={white},title={\begin{center}\texttt{Lean \thelcounter} \addtocounter{lcounter}{1} \end{center}},colbacktitle=Blue,coltitle=black] \begin{minted}[breaklines, escapeinside=||]{lean}
-- defining the Isbell adjunction on objects /-
-- Obj
-- def LanObj
-/
\end{minted} \end{tcolorbox} \end{center}
%LEAN:
\begin{center} \begin{tcolorbox}[width=5in,colback={white},title={\begin{center}\texttt{Lean \thelcounter} \addtocounter{lcounter}{1} \end{center}},colbacktitle=Blue,coltitle=black] \begin{minted}[breaklines, escapeinside=||]{lean}
-- defining the Isbell adjunction on morphisms /-
-/
\end{minted} \end{tcolorbox} \end{center}
%LEAN: proving the identity law for the Isbell adjunction
\begin{center} \begin{tcolorbox}[width=5in,colback={white},title={\begin{center}\texttt{Lean \thelcounter} \addtocounter{lcounter}{1} \end{center}},colbacktitle=Blue,coltitle=black] \begin{minted}[breaklines, escapeinside=||]{lean}
-- proving the identity law for the Isbell adjunction /-

 -/
\end{minted} \end{tcolorbox} \end{center}
%LEAN: proving compositionality for the Isbell adjunction
\begin{center} \begin{tcolorbox}[width=5in,colback={white},title={\begin{center}\texttt{Lean \thelcounter} \addtocounter{lcounter}{1} \end{center}},colbacktitle=Blue,coltitle=black] \begin{minted}[breaklines, escapeinside=||]{lean}
-- proving compositionality for the Isbell adjunction /-
-/
\end{minted} \end{tcolorbox} \end{center}
%LEAN: assembling the functor of the Isbell adjunction
\begin{center} \begin{tcolorbox}[width=5in,colback={white},title={\begin{center}\texttt{Lean \thelcounter} \addtocounter{lcounter}{1} \end{center}},colbacktitle=Blue,coltitle=black] \begin{minted}[breaklines, escapeinside=||]{lean}
-- assembling the functor of the Isbell adjunction /-
-/
\end{minted} \end{tcolorbox} \end{center}
%LEAN: unit of the Isbell adjunction on objects
\begin{center} \begin{tcolorbox}[width=5in,colback={white},title={\begin{center}\texttt{Lean \thelcounter} \addtocounter{lcounter}{1} \end{center}},colbacktitle=Blue,coltitle=black] \begin{minted}[breaklines, escapeinside=||]{lean}

 -- unit of the Isbell adjunction on objects /-
-/
\end{minted} \end{tcolorbox} \end{center}
%LEAN: unit of the Isbell adjunction naturality
\begin{center} \begin{tcolorbox}[width=5in,colback={white},title={\begin{center}\texttt{Lean \thelcounter} \addtocounter{lcounter}{1} \end{center}},colbacktitle=Blue,coltitle=black] \begin{minted}[breaklines, escapeinside=||]{lean}
-- unit of the Isbell adjunction naturality /-
-/
\end{minted} \end{tcolorbox} \end{center}
%LEAN: unit of the Isbell adjunction
\begin{center} \begin{tcolorbox}[width=5in,colback={white},title={\begin{center}\texttt{Lean \thelcounter} \addtocounter{lcounter}{1} \end{center}},colbacktitle=Blue,coltitle=black] \begin{minted}[breaklines, escapeinside=||]{lean}
-- unit of the Isbell adjunction /-
-/
\end{minted} \end{tcolorbox} \end{center}
%LEAN: triangle identity of the Isbell adjunction
\begin{center}

 \begin{tcolorbox}[width=5in,colback={white},title={\begin{center}\texttt{Lean \thelcounter} \addtocounter{lcounter}{1} \end{center}},colbacktitle=Blue,coltitle=black] \begin{minted}[breaklines, escapeinside=||]{lean}
-- triangle identity of the Isbell adjunction /-
-/
\end{minted} \end{tcolorbox} \end{center}
%LEAN: assembling the Isbell adjunction
\begin{center} \begin{tcolorbox}[width=5in,colback={white},title={\begin{center}\texttt{Lean \thelcounter} \addtocounter{lcounter}{1} \end{center}},colbacktitle=Blue,coltitle=black] \begin{minted}[breaklines, escapeinside=||]{lean}
-- assembling the Isbell adjunction /-
-/
\end{minted} \end{tcolorbox} \end{center}
\chapter{The Adjoint Functor Theorem}
{\footnotesize \begin{center} \begin{tabular}{|| l || l ||}
\hline
$\texttt{Section}$ & $\texttt{Description}$ \\
\hline
\hline
Right adjoints preserve limits & \\
\hline
Left adjoints preserve colimits & \\
\hline
The adjoint functor theorem for right adjoints & A theorem demonstrating the existence of a left adjoint \\

 \hline
The adjoint functor theorem for right adjoints & A theorem demonstrating the existence of a right adjoint \\
\hline
\end{tabular}
\end{center}} \\\
\\\
The adjoint functor theorem, also known as the adjoint functor existence theorem, was proved independently by three mathematicians: Daniel Kan, Saunders Mac Lane, and Samuel Eilenberg. The contribution of Kan came from his \href{https://www.ams.org/ journals/tran/1958-087-02/S0002-9947-1958-0131451-0/S0002-9947-1958-0131451-0.pdf } {mentioned paper} on adjoint functors, though it was not established in its full generality. This much appeared later in ``Categories for the Working Mathematician".\\
\section{Right adjoints preserve limits}
%LEAN: left adjoints preserve limits
\begin{center} \begin{tcolorbox}[width=5in,colback={white},title={\begin{center}\texttt{Lean \thelcounter} \addtocounter{lcounter}{1} \end{center}},colbacktitle=Blue,coltitle=black] \begin{minted}[breaklines, escapeinside=||]{lean}
-- left adjoints preserve limits /-
-/
\end{minted} \end{tcolorbox} \end{center}
\section{Left adjoints preserve colimits}
%LEAN: left adjoints preserve colimits
\begin{center} \begin{tcolorbox}[width=5in,colback={white},title={\begin{center}\texttt{Lean \thelcounter} \addtocounter{lcounter}{1} \end{center}},colbacktitle=Blue,coltitle=black] \begin{minted}[breaklines, escapeinside=||]{lean}
  
 -- left adjoints preserve colimits /-
-/
\end{minted} \end{tcolorbox} \end{center}
\section{The adjoint functor theorem for right adjoints}
%LEAN: definition of limit preservation
\begin{center} \begin{tcolorbox}[width=5in,colback={white},title={\begin{center}\texttt{Lean \thelcounter} \addtocounter{lcounter}{1} \end{center}},colbacktitle=Blue,coltitle=black] \begin{minted}[breaklines, escapeinside=||]{lean}
-- definition of limit preservation /-
-/
\end{minted} \end{tcolorbox} \end{center}
%LEAN: limit preserving functors are right adjoint
\begin{center} \begin{tcolorbox}[width=5in,colback={white},title={\begin{center}\texttt{Lean \thelcounter} \addtocounter{lcounter}{1} \end{center}},colbacktitle=Blue,coltitle=black] \begin{minted}[breaklines, escapeinside=||]{lean}
-- limit preserving functors are right adjoint /-
this should show that the map [Cop,Set] ⇄ [Dop,Set] -/
\end{minted}

 \end{tcolorbox} \end{center}
\section{The adjoint functor theorem for left adjoints}
%LEAN: definition of colimit preservation
\begin{center} \begin{tcolorbox}[width=5in,colback={white},title={\begin{center}\texttt{Lean \thelcounter} \addtocounter{lcounter}{1} \end{center}},colbacktitle=Blue,coltitle=black] \begin{minted}[breaklines, escapeinside=||]{lean}
-- definition of colimit preservation /-
-/
\end{minted} \end{tcolorbox} \end{center}
%LEAN: colimit preserving functors are left adjoint
\begin{center} \begin{tcolorbox}[width=5in,colback={white},title={\begin{center}\texttt{Lean \thelcounter} \addtocounter{lcounter}{1} \end{center}},colbacktitle=Blue,coltitle=black] \begin{minted}[breaklines, escapeinside=||]{lean}
-- colimit preserving functors are left adjoint /-
-/
\end{minted} \end{tcolorbox} \end{center}
\chapter{Colimits}
\iffalse
-- hocolim_G * is naturally the domain of a space over BG. -- we can define a map [B,∞] → [E,∞] which:
-- on objects sends F to f • (pullback with false of F)
-- on morphisms sends η to f • (pullback with false of η)

 - [Xop,∞] formally adds in ∞-colimits - homotopy Kan extensions too
\fi
\chapter{Pointed Kan Extensions}
\begin{enumerate}
\item Expressing a pointed Kan extension as a \end{enumerate}
https://arxiv.org/pdf/2207.07427.pdf https://arxiv.org/pdf/2212.03722.pdf https://arxiv.org/pdf/2109.12004.pdf https://arxiv.org/pdf/1705.09634.pdf https://arxiv.org/pdf/1705.09634.pdf https://arxiv.org/pdf/1905.13576.pdf https://www.jonathannilesweed.com/files/Wee18b.pdf
Reinhard Börger
\section{Some Mathematicians}
\begin{enumerate} \item Diophantus \item Evariste Galois \item Henri Poincare \item David Hilbert \item Leonhard Euler \item Andre Weil
 
 \item Alexandre Grothendieck \item Jean-Pierre Serre
\item John Tate
\item Pierre Deligne
\item Hermann Weyl \item Peter Dirichlet \item Michael Atiyah \item Atle Selberg \item Albert Einstein \item Joseph Fourier \item Ernst Kummer \item William Thurston \end{enumerate}




In this part we prove the Lefschetz fixed point theorem for reflexive loci using the approach to 

\begin{enumerate}[(a)]
\item $\texttt{ev (φॱ x ∪ y) = ev (x ∪ φ𛲔 y)}, where $\texttt{φ𛲔}$ is the covariant adjoint (cohomology being contravariant). This is the definition of $\texttt{φॱ}$. 
\item The chern class $\texttt{ch}$ is the image of $\texttt{1}$ under the adjoint $\texttt{φ𛲔}$
\item $\texttt{φ𛲔 • ψ𛲔 = (φ • ψ)𛲔}$
\item $\texttt{ev (φ𛲔 x) = ev x}$ (φ preserves cup with the euler class)
\item $\texttt{φ𛲔 (x ∪ φॱ y) = φ𛲔(x) ∪ y}$ (frobenius reciprocity)
\item $\texttt{φ𛲔 • φॱ}$ is multiplication by the degree of an extension when $\texttt{φ}$ is a finite map. 
\end{enumerate}

\begin{theorem}
Let $\texttt{X,Y: Loc.Obj}$, $\texttt{φ : X → Y}$ and $\texttt{Loc.Hom X × Y → X}$, $\texttt{Loc.Hom X × Y → Y}$. then p𛲔((cl (graph φ) (X × Y)) ∪ qॱ y) = φॱ y
\begin{align*}
= & $\texttt{p𛲔((cl (graph φ) (X × Y)) ∪ qॱy) = φॱy}$\\
\stackrel{(b)}{=} & $\texttt{ p𛲔((1 × φ)𛲔(1) ∪ qॱy) }$\\
\stackrel{(e)}{=} & \texttt{p𛲔 ∘ (1 × φ)𛲔 ∘ (1 ∪ (q ∘ (1 × φ))ॱ y)}\\
= & \texttt{(p ∘ (1 × φ))𛲔 ∘ (1 ∪ (q ∘ (1 × φ))ॱ y)}\\
= & \texttt{id𛲔(1 ∪ φॱ y)}\\
= & \texttt{φॱ y}
\end{align*}
\end{theorem}

\begin{enumerate}
\item $\texttt{cl\_(X × X) (Γ\_(φ) ×\_() Δ\_(τ\_(X)))}$ is trace in the top rung (something like $\texttt{e ∪ -}$)
\item (ev\_X × X) cl X × X (graph φ * diagonal X) is the number of fixed points of φ
\item $\texttt{p𛲔(cl X × Y (graph (φ)) ∪ qॱy) = φॱ(y)}$.
\item $\texttt{\# graph\_(φ) ∪ (diagonal X) = 𝕋𝕣 φ}$. 
\end{enumerate}


\iffalse
- matrix notation for biproducts
- kernel: pullback with  0 -> M
- cokernel: pushout with N -> 0
- coimage: cokernel of kernel
- image: kernel of cokernel (in groups, the coimage and image are distinct.)

- there is an idempotent adjunction: cok: C / N <-> C \ N :ker
- corollary: ker(M -> M) = 0 -> M, cok(M -> M) = M -> 0, kernel of pullback, cokernel of pushout
- equivalence of normals, conormals

- N -f-> M -g-> L is exact if im(f) ~ ker(g), equivalently cok(f) ~ coim(g)
- same for chains
- image-coimage factorization: dom(f) -> im(f) -> coim(f) -> cod(f) is exact
- (L, M, N, P) is a pullation square if and only if 0 -> L -> M oplus N -> P -> 0 is exact.

- the first isomorphism theorem: coim -> im is an isomorphism
- the relationship between the first isomorphism theorem and frobenius reciprocity

For N -> L -> M, we have
\begin{enumerate}[(a)]
\item $0 -> L/N -> M/N -> M / L -> 0$ is exact
\item $0 -> cok(f) -> cok(fg) -> cok(g) -> 0$
\item $0 -> M/(N cap L) = (M+N)/L -> 0$ is exact
\item $0 -> M/(N cap L) -> M/N oplus M/L -> M / N+L -> 0$ is exact (Meyer Vietoris)
\item if N + L = M,  then M/(N cap L) -> M/N oplus M/L is an isomorphism by the first isomorphism theorem.
\item if N cap L = 0,  then M/N oplus M/L -> M/N oplus M/L is an isomorphism by the first isomorphism theorem.
\item $0 -> M/(ca∃_i N_i) -> oplus M/N_i -> 0$ is exact if $N_i + N_j = 1$ for each i, j.
\item there is an idempotent adjunction: cok: C / N <-> C \ N :ker
\item corollary: ker(M -> M) = 0 -> M, cok(M -> M) = M -> 0, kernel of pullback, cokernel of pushout
\item equivalence of normals, conormals
\item $N -f-> M -g-> L$ is exact if $im(f) \sim ker(g)$, equivalently $cok(f) \sim coim(g)$
\item same for chains
\item $(L, M, N, P)$ is a pullation square if and only if $0 -> L -> M$ oplus $N -> P -> 0$ is exact.
\end{enumerate}


- abelian: has a coimage image isomorphism theorem.
- define saturated: 0 -> M -> N -> 0 exact implies M -> N is an iso.
- how point lifting relates to the abelian condition
- monos, epis, comonos, coepis, that these are equivalent to abelian.
- subtracting morphisms given the abelian condition
- splitting lemma: let 0 -> N -> M -> L -> 0 be an exact sequence. Then the following are equivalent:
1) The sequence has a section
2) The sequence has a retract
3) The sequence is canonically a biproduct
\fi

The Tempest, by William Shakespeare
Collected Speeches, by Abraham Lincoln
A River Runs Through It, by Norman MacLean












\chapter{Chapter 4: The Group Fixed Point Principals}

\section{\texttt{B}}

$\texttt{B}$ is the ordinary classifying space, and it is defined on internal groups in D(∞-Grpd₀). 

%LEAN: B
\begin{center}
\begin{tcolorbox}[width=5in,colback={white},title={\begin{center}\texttt{Lean \thelcounter} \addtocounter{lcounter}{1}  \end{center}},colbacktitle=Blue,coltitle=black]
\begin{minted}[breaklines, escapeinside=||]{lean}

-- def BInfGrpd : (𝐂𝐚𝐭.Hom Grpd_(∞-𝔾𝕣𝕡𝕕) D(∞-𝔾𝕣𝕡𝕕)).Obj := sorry

\end{minted}
\end{tcolorbox}
\end{center}

\section{\texttt{b}}

$\texttt{B}$ is the ordinary classifying space, and it is defined on internal group actions in D(∞-Grpd₀). 

%LEAN: b
\begin{center}
\begin{tcolorbox}[width=5in,colback={white},title={\begin{center}\texttt{Lean \thelcounter} \addtocounter{lcounter}{1}  \end{center}},colbacktitle=Blue,coltitle=black]
\begin{minted}[breaklines, escapeinside=||]{lean}

-- def Par (C : D(∞-ℂ𝕒𝕥).Obj) : Shf_(∞-ℂ𝕒𝕥) (P_(∞-ℂ𝕒𝕥) C C (𝟙_(D(∞-ℂ𝕒𝕥)) C)) ⭢ (Cmp_(∞-ℂ𝕒𝕥) C) := sorry

\end{minted}
\end{tcolorbox}
\end{center}

%LEAN: 
\begin{center}
\begin{tcolorbox}[width=5in,colback={white},title={\begin{center}\texttt{Lean \thelcounter} \addtocounter{lcounter}{1}  \end{center}},colbacktitle=Blue,coltitle=black]
\begin{minted}[breaklines, escapeinside=||]{lean}

-- notation "b" => Par

\end{minted}
\end{tcolorbox}
\end{center}

\section{The Internal Group Fixed Point Principal}

For a based connected space X, the path space [I,X] is weak equivalent to the loop space ΩX. This observation will allow us to prove that the category of based connected ∞-groupoids is internal groups in itself.\\

%LEAN: 
\begin{center}
\begin{tcolorbox}[width=5in,colback={white},title={\begin{center}\texttt{Lean \thelcounter} \addtocounter{lcounter}{1}  \end{center}},colbacktitle=Yellow,coltitle=black]
\begin{minted}[breaklines, escapeinside=||]{lean}

def internal_groupoid_delooping_principal (Γ : pulback_system) : Type := D(Γ) ≃_(𝐂𝐚𝐭) Grpd_(Γ)

\end{minted}
\end{tcolorbox}
\end{center}


%LEAN: Id₁Fst
\begin{center}
\begin{tcolorbox}[width=5in,colback={white},title={\begin{center}\texttt{Lean \thelcounter} \addtocounter{lcounter}{1}  \end{center}},colbacktitle=Blue,coltitle=black]
\begin{minted}[breaklines, escapeinside=||]{lean}

\end{minted}
\end{tcolorbox}
\end{center}

%LEAN: Id₁Snd
\begin{center}
\begin{tcolorbox}[width=5in,colback={white},title={\begin{center}\texttt{Lean \thelcounter} \addtocounter{lcounter}{1}  \end{center}},colbacktitle=Blue,coltitle=black]
\begin{minted}[breaklines, escapeinside=||]{lean}

\end{minted}
\end{tcolorbox}
\end{center}

%LEAN: Id₁Id₁
\begin{center}
\begin{tcolorbox}[width=5in,colback={white},title={\begin{center}\texttt{Lean \thelcounter} \addtocounter{lcounter}{1}  \end{center}},colbacktitle=Blue,coltitle=black]
\begin{minted}[breaklines, escapeinside=||]{lean}

\end{minted}
\end{tcolorbox}
\end{center}

%LEAN: Id₁Id₂
\begin{center}
\begin{tcolorbox}[width=5in,colback={white},title={\begin{center}\texttt{Lean \thelcounter} \addtocounter{lcounter}{1}  \end{center}},colbacktitle=Blue,coltitle=black]
\begin{minted}[breaklines, escapeinside=||]{lean}

\end{minted}
\end{tcolorbox}
\end{center}

%LEAN: Id₁
\begin{center}
\begin{tcolorbox}[width=5in,colback={white},title={\begin{center}\texttt{Lean \thelcounter} \addtocounter{lcounter}{1}  \end{center}},colbacktitle=Blue,coltitle=black]
\begin{minted}[breaklines, escapeinside=||]{lean}

\end{minted}
\end{tcolorbox}
\end{center}

%LEAN: Id₂Fst
\begin{center}
\begin{tcolorbox}[width=5in,colback={white},title={\begin{center}\texttt{Lean \thelcounter} \addtocounter{lcounter}{1}  \end{center}},colbacktitle=Blue,coltitle=black]
\begin{minted}[breaklines, escapeinside=||]{lean}

\end{minted}
\end{tcolorbox}
\end{center}

%LEAN: Id₂Snd
\begin{center}
\begin{tcolorbox}[width=5in,colback={white},title={\begin{center}\texttt{Lean \thelcounter} \addtocounter{lcounter}{1}  \end{center}},colbacktitle=Blue,coltitle=black]
\begin{minted}[breaklines, escapeinside=||]{lean}

\end{minted}
\end{tcolorbox}
\end{center}

%LEAN: Id₂Id₁
\begin{center}
\begin{tcolorbox}[width=5in,colback={white},title={\begin{center}\texttt{Lean \thelcounter} \addtocounter{lcounter}{1}  \end{center}},colbacktitle=Blue,coltitle=black]
\begin{minted}[breaklines, escapeinside=||]{lean}

\end{minted}
\end{tcolorbox}
\end{center}

%LEAN: Id₂Id₂
\begin{center}
\begin{tcolorbox}[width=5in,colback={white},title={\begin{center}\texttt{Lean \thelcounter} \addtocounter{lcounter}{1}  \end{center}},colbacktitle=Blue,coltitle=black]
\begin{minted}[breaklines, escapeinside=||]{lean}

\end{minted}
\end{tcolorbox}
\end{center}

%LEAN: Id₂
\begin{center}
\begin{tcolorbox}[width=5in,colback={white},title={\begin{center}\texttt{Lean \thelcounter} \addtocounter{lcounter}{1}  \end{center}},colbacktitle=Blue,coltitle=black]
\begin{minted}[breaklines, escapeinside=||]{lean}

-- def internal_category_delooping_principal_proofId₂ : 

\end{minted}
\end{tcolorbox}
\end{center}

%LEAN: 
\begin{center}
\begin{tcolorbox}[width=5in,colback={white},title={\begin{center}\texttt{Lean \thelcounter} \addtocounter{lcounter}{1}  \end{center}},colbacktitle=Yellow,coltitle=black]
\begin{minted}[breaklines, escapeinside=||]{lean}

-- def internal_category_delooping_principal_proof : internal_category_delooping_principal ∞-ℂ𝕒𝕥 := sorry

\end{minted}
\end{tcolorbox}
\end{center}


\section{The Internal Group Action Fixed Point Principal}

For a based connected space X, a based connected space Y, and a based map f : X ⭢ Y, the homotopy pullback of f with 𝟙 Y is weak equivalent the homotopy pullback with the base. This fascinating insight 

%LEAN: 
\begin{center}
\begin{tcolorbox}[width=5in,colback={white},title={\begin{center}\texttt{Lean \thelcounter} \addtocounter{lcounter}{1}  \end{center}},colbacktitle=Yellow,coltitle=black]
\begin{minted}[breaklines, escapeinside=||]{lean}

def internal_groupoid_action_delooping_principal (Γ : pullback_system) (C : D(Γ).Obj) : Type := Shf_(Γ) (P_(Γ) C C (𝟙_(D(Γ)) C)) ≃_(𝐂𝐚𝐭) Der_(Γ) C

\end{minted}
\end{tcolorbox}
\end{center}

%LEAN: Id₁Fst
\begin{center}
\begin{tcolorbox}[width=5in,colback={white},title={\begin{center}\texttt{Lean \thelcounter} \addtocounter{lcounter}{1}  \end{center}},colbacktitle=Blue,coltitle=black]
\begin{minted}[breaklines, escapeinside=||]{lean}

\end{minted}
\end{tcolorbox}
\end{center}

%LEAN: Id₁Snd
\begin{center}
\begin{tcolorbox}[width=5in,colback={white},title={\begin{center}\texttt{Lean \thelcounter} \addtocounter{lcounter}{1}  \end{center}},colbacktitle=Blue,coltitle=black]
\begin{minted}[breaklines, escapeinside=||]{lean}

\end{minted}
\end{tcolorbox}
\end{center}

%LEAN: Id₁Id₁
\begin{center}
\begin{tcolorbox}[width=5in,colback={white},title={\begin{center}\texttt{Lean \thelcounter} \addtocounter{lcounter}{1}  \end{center}},colbacktitle=Blue,coltitle=black]
\begin{minted}[breaklines, escapeinside=||]{lean}

\end{minted}
\end{tcolorbox}
\end{center}

%LEAN: Id₁Id₂
\begin{center}
\begin{tcolorbox}[width=5in,colback={white},title={\begin{center}\texttt{Lean \thelcounter} \addtocounter{lcounter}{1}  \end{center}},colbacktitle=Blue,coltitle=black]
\begin{minted}[breaklines, escapeinside=||]{lean}

\end{minted}
\end{tcolorbox}
\end{center}

%LEAN: Id₁
\begin{center}
\begin{tcolorbox}[width=5in,colback={white},title={\begin{center}\texttt{Lean \thelcounter} \addtocounter{lcounter}{1}  \end{center}},colbacktitle=Blue,coltitle=black]
\begin{minted}[breaklines, escapeinside=||]{lean}

\end{minted}
\end{tcolorbox}
\end{center}

%LEAN: Id₂Fst
\begin{center}
\begin{tcolorbox}[width=5in,colback={white},title={\begin{center}\texttt{Lean \thelcounter} \addtocounter{lcounter}{1}  \end{center}},colbacktitle=Blue,coltitle=black]
\begin{minted}[breaklines, escapeinside=||]{lean}

\end{minted}
\end{tcolorbox}
\end{center}

%LEAN: Id₂Snd
\begin{center}
\begin{tcolorbox}[width=5in,colback={white},title={\begin{center}\texttt{Lean \thelcounter} \addtocounter{lcounter}{1}  \end{center}},colbacktitle=Blue,coltitle=black]
\begin{minted}[breaklines, escapeinside=||]{lean}

\end{minted}
\end{tcolorbox}
\end{center}

%LEAN: Id₂Id₁
\begin{center}
\begin{tcolorbox}[width=5in,colback={white},title={\begin{center}\texttt{Lean \thelcounter} \addtocounter{lcounter}{1}  \end{center}},colbacktitle=Blue,coltitle=black]
\begin{minted}[breaklines, escapeinside=||]{lean}

\end{minted}
\end{tcolorbox}
\end{center}

%LEAN: Id₂Id₂
\begin{center}
\begin{tcolorbox}[width=5in,colback={white},title={\begin{center}\texttt{Lean \thelcounter} \addtocounter{lcounter}{1}  \end{center}},colbacktitle=Blue,coltitle=black]
\begin{minted}[breaklines, escapeinside=||]{lean}

\end{minted}
\end{tcolorbox}
\end{center}

%LEAN: Id₂
\begin{center}
\begin{tcolorbox}[width=5in,colback={white},title={\begin{center}\texttt{Lean \thelcounter} \addtocounter{lcounter}{1}  \end{center}},colbacktitle=Blue,coltitle=black]
\begin{minted}[breaklines, escapeinside=||]{lean}

\end{minted}
\end{tcolorbox}
\end{center}


%LEAN: 
\begin{center}
\begin{tcolorbox}[width=5in,colback={white},title={\begin{center}\texttt{Lean \thelcounter} \addtocounter{lcounter}{1}  \end{center}},colbacktitle=Yellow,coltitle=black]
\begin{minted}[breaklines, escapeinside=||]{lean}

-- def internal_groupoid_action_delooping_principal_proof (C : D(∞-ℂ𝕒𝕥).Obj) : internal_presheaf_delooping_principal ∞-ℂ𝕒𝕥 C  := sorry

\end{minted}
\end{tcolorbox}
\end{center}

%LEAN: 
\begin{center}
\begin{tcolorbox}[width=5in,colback={white},title={\begin{center}\texttt{Lean \thelcounter} \addtocounter{lcounter}{1}  \end{center}},colbacktitle=Blue,coltitle=black]
\begin{minted}[breaklines, escapeinside=||]{lean}



\end{minted}
\end{tcolorbox}
\end{center}




\chapter{Chapter 17: The Puppe Sequence for ∞-Categories}

In this chapter we construct the Puppe sequence for π⃗ₙ. {\bf Note: one joint in this exact sequence consists not of a map but an action.\} This will be used in the next chapter two establish two of the six categorical equivalences.\\


\chapter{Chapter 18: The Categorical Equivalences Involving B⃗ and b⃗}

After the construction in chapter 11, we will prove the internal category delooping principal, which is the first categorical equivalence of the six mentioned in the introduction. We also prove in this chapter the internal C-presheaf delooping principal, which is the second categorical equivalence of the six mentioned in the introduction. To do this, we first define B⃗ = B⃗\_(∞-ℂ𝕒𝕥) and b⃗ = b⃗\_(∞-ℂ𝕒𝕥).\\

This much may be possible for the case of simplicial sets using first the construction of E⃗ as a directed homotopy colimit (we can use Mathlib's geometric realization), and then quotienting by an apparant action of a particular internal category.\\

\section{\texttt{B⃗}}

%LEAN: B⃗
\begin{center}
\begin{tcolorbox}[width=5in,colback={white},title={\begin{center}\texttt{Lean \thelcounter} \addtocounter{lcounter}{1}  \end{center}},colbacktitle=Blue,coltitle=black]
\begin{minted}[breaklines, escapeinside=||]{lean}

-- def B⃗ : (𝐂𝐚𝐭.Hom Cat_(∞-ℂ𝕒𝕥) D(∞-ℂ𝕒𝕥)).Obj := sorry

-- 

\end{minted}
\end{tcolorbox}
\end{center}

\section{\texttt{b⃗}}

The $\texttt{b⃗}$ symbol formally gives a pseudofunctor, but we can also create a model in which it is a functor. It occurs as one side of a categorical equivalence, the second of the six categorical equivalences called ``delooping principals".\\

%LEAN: b⃗
\begin{center}
\begin{tcolorbox}[width=5in,colback={white},title={\begin{center}\texttt{Lean \thelcounter} \addtocounter{lcounter}{1}  \end{center}},colbacktitle=Blue,coltitle=black]
\begin{minted}[breaklines, escapeinside=||]{lean}

-- def Par (C : D(∞-ℂ𝕒𝕥).Obj) : Shf_(∞-ℂ𝕒𝕥) (P_(∞-ℂ𝕒𝕥) C C (𝟙_(D(∞-ℂ𝕒𝕥)) C)) ⭢ (Cmp_(∞-ℂ𝕒𝕥) C) := sorry

\end{minted}
\end{tcolorbox}
\end{center}

%LEAN: 
\begin{center}
\begin{tcolorbox}[width=5in,colback={white},title={\begin{center}\texttt{Lean \thelcounter} \addtocounter{lcounter}{1}  \end{center}},colbacktitle=Blue,coltitle=black]
\begin{minted}[breaklines, escapeinside=||]{lean}

-- notation "b⃗" => Par

\end{minted}
\end{tcolorbox}
\end{center}

\section{The B⃗-P⃗ Equivalence}

The internal category delooping principal will look something like this:

%LEAN: 
\begin{center}
\begin{tcolorbox}[width=5in,colback={white},title={\begin{center}\texttt{Lean \thelcounter} \addtocounter{lcounter}{1}  \end{center}},colbacktitle=Yellow,coltitle=black]
\begin{minted}[breaklines, escapeinside=||]{lean}

-- def internal_category_delooping_principal : Type := D(∞-ℂ𝕒𝕥) ≃ (DeloopableIntCat D(∞-ℂ𝕒𝕥))

\end{minted}
\end{tcolorbox}
\end{center}

It should be readily available from the construction in the last chapter.\\

%LEAN: 
\begin{center}
\begin{tcolorbox}[width=5in,colback={white},title={\begin{center}\texttt{Lean \thelcounter} \addtocounter{lcounter}{1}  \end{center}},colbacktitle=Yellow,coltitle=black]
\begin{minted}[breaklines, escapeinside=||]{lean}

/-
-- def internal_category_delooping_principal_proof : internal_category_delooping_principal := {Fst := internal_category_delooping_principalFst, Snd := internal_category_delooping_principalSnd, Id₁ := internal_category_delooping_principalId₁, Id₂ := internal_category_delooping_principalId₂}
-/

\end{minted}
\end{tcolorbox}
\end{center}

\section{The b⃗-p⃗ Equivalence}

The internal presheaf delooping principal consists of a categorical equivalence between D(∞-Cat/C) and internal $C$-presheaves in D(∞-Cat/C).\\

%LEAN: 
\begin{center}
\begin{tcolorbox}[width=5in,colback={white},title={\begin{center}\texttt{Lean \thelcounter} \addtocounter{lcounter}{1}  \end{center}},colbacktitle=Yellow,coltitle=black]
\begin{minted}[breaklines, escapeinside=||]{lean}

/-
def internal_presheaf_delooping_principal (C : D(∞-ℂ𝕒𝕥).Obj) : Type := Shf_(∞-ℂ𝕒𝕥) (P_(∞-ℂ𝕒𝕥) C C (𝟙_(D(∞-ℂ𝕒𝕥)) C)) ≃_(𝐂𝐚𝐭) (!_(𝐂𝐚𝐭) (?_(𝐂𝐚𝐭) (¡_(𝐂𝐚𝐭) (¿_(𝐂𝐚𝐭) (p_(∞-ℂ𝕒𝕥) C C (𝟙_(D(∞-ℂ𝕒𝕥)) C)))))).Cod
-/

\end{minted}
\end{tcolorbox}
\end{center}

Next we prove the internal C-sheaf delooping principal. This says says that \texttt{Shf\_(∞-ℂ𝕒𝕥) (P\_(∞-ℂ𝕒𝕥) C C (𝟙\_(D(∞-ℂ𝕒𝕥)) C)) ≃\_(𝐂𝐚𝐭) !? (p\_(∞-ℂ𝕒𝕥) C C (𝟙\_(D(∞-ℂ𝕒𝕥)) C))}.\\

%LEAN: The internal C-sheaf delooping principal
\begin{center}
\begin{tcolorbox}[width=5in,colback={white},title={\begin{center}\texttt{Lean \thelcounter} \addtocounter{lcounter}{1}  \end{center}},colbacktitle=Yellow,coltitle=black]
\begin{minted}[breaklines, escapeinside=||]{lean}

-- The internal C-sheaf delooping principal
/-
def internal_presheaf_delooping_principal_proof (C : D(∞-ℂ𝕒𝕥).Obj) : internal_presheaf_delooping_principal C := {Fst := internal_presheaf_delooping_principalFst, Snd := internal_presheaf_delooping_principalSnd, Id₁ := internal_presheaf_delooping_principalId₁, Id₂ := internal_presheaf_delooping_principalId₂}
-/

\end{minted}
\end{tcolorbox}
\end{center}
\fi




\newpage 
\ \\
\ \\
\ \\
\ \\
\ \\
\ \\
%LEAN: 
\begin{center}
\begin{tcolorbox}[width=5in,colback={white},title={\begin{center}\texttt{About the Author} \addtocounter{lcounter}{1}  \end{center}},colbacktitle=Yellow,coltitle=black]
Dean Young is a graduate student at New York University, where he studies mathematics. \\

\begin{center}
\iffalse \includegraphics[width=7.5cm,height=5cm]{about.jpg}\fi 
\end{center}
\end{tcolorbox}
\end{center}
\newpage
\ \\
\thispagestyle{empty}
\pagecolor{Yellow}



\end{document}







































